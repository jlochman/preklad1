\clearpage
\section{Equipment No.09.05-1	- Office area}

\twoColls{
  Větrání prostor kanceláří a jednacích místnosti zajišťuje jedna vzduchotechnická
  jednotka osazená ve strojovně vzduchotechniky v 2NP. Teplotně upravovaný
  venkovní vzduch je přiváděn potrubím do vnitřních prostor pomocí vířivých
  anemostatů, jež jsou osazeny do podhledu, a znehodnocený vzduch je odsáván
  potrubím vedeným nad podhledem. Propojení mezi větranou místností a odsávaným
  podhledem, bude zajištěno mřížovanou podhledovou kazetou. 

  Pro každou větranou
  zónu (kancelář, zasedací místnost, archiv, čajová kuchyňka), budou do přívodního
  potrubí osazeny regulátory průtoku VAV s vodními ohřívači a do odvodního potrubí
  regulátory průtoku VAV. Regulátory průtoku VAV s ohřívači na přívodu budou
  regulovány dle prostorového teplotního čidla z každé místnosti. Regulátory VAV
  na odvodu budou regulovány souběžně s regulátory VAV na přívodu. Otáčky
  ventilátorů v centrálních jednotkách budou řízeny na konstantní tlak v potrubí.

}{
  Office areas and meeting rooms will be ventilated by single AHU which will be
  placed in AC machinery room on the 2nd floor.  Thermally conditioned outdoor air
  will be supplied through ventilation duct to indoor areas by swirl diffusers,
  which will be installed in ceiling.  Waste air will be exhausted by ducts
  leading above the ceiling.  Ventilated office will be interconnected with
  exhausting ceiling by reticulated cartridges mounted in the ceiling. 

  For each ventilated area (office, meeting room, management archive or tea
  kitchen), supply ducts will be equipped with flow rate VAV controllers with
  water-based air heaters, whereas the exhaust ducts will be equipped with flow
  rate VAV controllers only. Whole equipment in inlets will be controlled
  according to temperature sensor in each room. As soon as these controllers are
  set, controller settings on outlets will be determined.  Fan speeds in central
  AHUs will be set to ensure constant pressure in supply and exhaust ducts.

}

\twoColls{
  V každé místnosti bude umístěno čidlo kvality vzduchu, a poměr směšování
  v centrální VZT jednotce, bude řízen dle nejhorší hodnoty ze všech čidel.
  Přiváděná dávka venkovního vzduchu do kanceláře je 50m3/h na osobu a do zasedací
  místnosti 80m3/h na osobu, nebo min. 4x/h.

}{
  Air quality sensor will be placed in each room and according to the worse
  sensor value, the mixing ration of fresh air will be set in central AHU. The
  amount of outdoor air supplied to offices is calculated for 50~m3/h per person
  and for meeting rooms for 80~m3/h per person. There will always be at least 4
  air exchanges per hour.

}

\twoColls{
  Vzduchotechnická jednotka je ve vnitřním provedení, opatřena filtrem vzduchu
  (přívod F5+F7 /odtah F5), rotačním výměníkem, směšovací komorou, vodním
  chladičem (15/20\gc), vodním chladičem (9/15\gc), vodním ohřívačem (45/30\gc) a
  dvěma ventilátory s frekvenčními měniči. 

  Zařízení pracuje s nuceným přívodem a
  odvodem vzduchu. Čerstvý vzduch je nasáván VZT jednotkou potrubím z fasády
  objektu, kde je umístěna protidešťová žaluzie. Znehodnocený vzduch je VZT
  jednotkou vyfukován nad střechu objektu.       

}{
  Innerly designed AHU will be equipped with air filters (F5+F7 supply / F5
  exhaust), rotatory heat exchanger, mixing chamber, water cooler (15 / 20\gc),
  water cooler (9 / 15\gc), water heater (45 / 30\gc) and two fans with frequency
  changers.

  Equipment will work with forced air inlet and outlet. Fresh air will be
  suctioned in by AHU through ducts from building facade where a weather
  resistant louver will be placed. Waste air will be blown by AHU over the roof
  of the building.

}

\twoColls{
  Chlazení kanceláří je zajištěno pomocí centrální VZT jednotky (VAV systém).
  Maximální přiváděné množství vzduchu do kanceláře a zasedací místnosti je
  navrženo na odvod letní tepelné zátěže, snížení množství přívodního vzduchu
  zajistí minimální hygienické větrání. V zasedací místnosti bude zajištěna
  možnost zcela uzavřít přívod a odvod vzduchu (zajistí profese MaR) tlačítkem
  z prostoru místnosti. 

}{
  Offices cooling will be ensured by central AHU (variable air volume system –
  VAV). Maximal amount of air supplied to offices and meeting room is designed to
  remove heat gains in summer time. Amount of supplied air will always comply with
  a requirement of minimal hygienic ventilation.  In meeting room, there is a
  possibility of complete closure of the ventilation systems by the switches
  (ensured by profession M\&C – Measurement and Control). 

}

\twoColls{
  Návrhové parametry jednotky, jsou uvedeny v příloze: TABLE EQUIPMENT AND
  PRINCIPLE DIAGRAMS.

}{
  Unit design parameters are listed in annex: TABLE EQUIPMENT AND PRINCIPLE
  DIAGRAMS.

}

\clearpage
\section{Equipment No.09.05-2	- Caffeteria}

\twoColls{
  Větrání, chlazení a vytápění jídelny zajišťuje jedna vzduchotechnická jednotka
  osazená ve strojovně vzduchotechniky v 2NP. Teplotně upravovaný venkovní vzduch
  je přiváděn potrubím do prostoru, kam je vyfukován pomocí vířivých vyústek. 

}{
  Ventilation, cooling and heating of Caffeteria will be ensured by single AHU
  placed in AC machinery room on the 2nd floor. Thermally conditioned outdoor air
  will be forced through ventilation duct to caffeteria, where it will be
  blown out through swirl diffusers. 

}

\twoColls{
  Odvod vzduchu je pomocí dvou potrubních tras vedených nad podhledem.  Jedno
  odsávací potrubí je místně vyústěno nad podhledem a přisávání vzduchu je
  zajištěno přes mřížované podhledové kazety.  Druhé odsávací potrubí je vedeno
  nad výdejem jídel a do potrubí jsou napojeny odsávací vyústky-odlučovače tuku
  osazené do podhledu. 

}{

}

\twoColls{
  Množství přiváděného vzduchu je navrženo na 10 výměn vzduchu za hodinu, tak aby
  byla odvedena tepelná zátěž z prostru. Při přiváděném množství vzduchu
  13.500m3/h a počtu 180 osob, potom vychází dávka venkovního vzduchu
  75m3/h/osobu.

}{

}

\twoColls{
  Vzduchotechnická jednotka je ve vnitřním provedení, opatřena filtrem vzduchu
  (přívod F5+F7 /odtah GREASE FILTER+F5), deskovým výměníkem s obtokem, vodním
  chladičem (15/20\gc), vodním chladičem (9/15\gc), vodním ohřívačem (45/30\gc) a
  dvěma ventilátory s frekvenčními měniči. 

  Zařízení pracuje s nuceným přívodem a
  odvodem vzduchu. Čerstvý vzduch je nasáván VZT jednotkou potrubím z fasády
  objektu, kde je umístěna protidešťová žaluzie. Znehodnocený vzduch je VZT
  jednotkou vyfukován nad střechu objektu.       

}{
  Innerly designed AHU will be equipped with air filters (F5+F7 supply /
  F5+GREASE FILTER exhaust), plate heat exchanger with bypass, water-based air
  cooler (15 / 20\gc), water-based air cooler (9 / 15\gc) water-based air
  heater (45 / 30\gc) and two fans with frequency changers. 

  Equipment will work with forced air inlet and outlet. Fresh air will be
  suctioned in by AHU through ducts from building facade, where a weather
  resistant louver will be placed.  Waste air will be blown by AHU over the roof
  of the building.

}

\twoColls{
  Návrhové parametry jednotky, jsou uvedeny v příloze: TABLE EQUIPMENT AND
  PRINCIPLE DIAGRAMS.

}{
  Unit design parameters are listed in annex: TABLE EQUIPMENT AND PRINCIPLE
  DIAGRAMS.

}


\clearpage
\section{Equipment No.09.05-3	- Kitchen}

\twoColls{
Větrání, chlazení a vytápění prostor kuchyně zajišťují dvě vzduchotechnické
jednotky osazené ve strojovně vzduchotechniky v 2NP. Jednotka poz. 3.1 zajišťuje
větrání varny, studené kuchyně (cold kitchen) a dvou umýváren nádobí. Jednotka
poz. 3.2 zajišťuje větrání menších místností náležících k zázemí kuchyně (šatny,
kanceláře, chodby, sklady atd.).

}{

}

\twoColls{
Jednotkou poz. 3.1 jsou větrány místnosti KITCHEN, COLD KITCHEN a 2x WASHING. 

}{

}

\twoColls{
Teplotně upravovaný venkovní vzduch je přiváděn potrubím do vnitřních prostor
kuchyně z větší části pomocí velkoplošných vyústek integrovaných do odsávacích
zákrytů a z menší části pomocí vířivých anemostatů, jež jsou osazeny do
podhledu. Do studené kuchyně a umýváren je upravený vzduch přiváděn pomocí
vířivých anemostatů, jež jsou osazeny do podhledu.

}{

}

\twoColls{
Znehodnocený vzduch je odsáván z kuchyně a mytí nádobí z větší části pomocí
odsávacích zákrytů a z menší části pomocí vyústek-odlučovačů tuků osazených do
podhledů. 

}{

}

\twoColls{
V kuchyni budou ve středu místnosti nad varnými spotřebiči umístěny nerezové
odsávací digestoře HALTON s integrovanými velkoplošnými vyústkami pro přívod
vzduchu. Obě velké digestoře budou vybaveny hasícím systémem ANSUL. Boxy ANSUL
budou osazeny na boku digestoře. Nad konvektomatem bude umístěna nerezová
odsávací digestoř HALTON s integrovanými velkoplošnými vyústkami pro přívod
vzduchu. Nad myčkami v místnostech mytí nádobí, budou umístěny kondenzační
nerezové odsávací digestoře HALTON.

}{

}

\twoColls{
Vzduchotechnická jednotka je ve vnitřním provedení, opatřena filtrem vzduchu
(přívod F5+F7 /odtah GREASE FILTER+F5), deskovým výměníkem s obtokem, vodním
chladičem (15/20\gc), vodním chladičem (9/15\gc), vodním ohřívačem (45/30\gc) a
dvěma ventilátory s frekvenčními měniči. 

Zařízení pracuje s nuceným přívodem a odvodem vzduchu. Čerstvý vzduch je nasáván
VZT jednotkou potrubím z fasády objektu, kde je umístěna protidešťová žaluzie.
Znehodnocený vzduch je VZT jednotkou vyfukován nad střechu objektu.       

}{

}

\twoColls{
Návrhové parametry jednotky, jsou uvedeny v příloze: TABLE EQUIPMENT AND
PRINCIPLE DIAGRAMS.

}{
Unit design parameters are listed in annex: TABLE EQUIPMENT AND PRINCIPLE
DIAGRAMS.

}

\twoColls{
Jednotkou poz. 3.2, jsou větrány místnosti OFFICE, CORRIDOR, CLOAK ROOM SHOWER,
STORAGE, WASHING THERMOPORTS.   

}{

}

\twoColls{
Teplotně upravovaný venkovní vzduch je přiváděn potrubím do vnitřních prostor
pomocí vířivých anemostatů, jež jsou osazeny do podhledu. Znehodnocený vzduch je
odsáván potrubím vedeným nad podhledem. Propojení mezi větranou místností a
odsávaným podhledem, bude zajištěno mřížovanou podhledovou kazetou. 

}{

}

\twoColls{
Jelikož, je jedním centrálním zařízením klimatizováno více prostor, jsou
v přívodním potrubí do kanceláří a šaten osazeny regulátory průtoku VAV
s vodními ohřívači a v odvodním potrubí regulátory průtoku VAV. Regulátory
průtoku VAV s ohřívači na přívodu budou regulovány dle prostorového teplotního
čidla z každé místnosti. 

Pro ostatní prostory jsou skupinově do přívodního a
odsávacího potrubí osazeny pouze regulátory VAV. Regulátory VAV na odvodu budou
regulovány souběžně s regulátory VAV na přívodu. Otáčky ventilátorů
v centrálních jednotkách budou řízeny na konstantní tlak v potrubí. Množství
přiváděného vzduchu je navrženo dle standardu LEGO.  

}{

}

\twoColls{
Vzduchotechnická jednotka je ve vnitřním provedení, opatřena filtrem vzduchu
(přívod F5+F7 /odtah F5), deskovým výměníkem s obtokem, vodním chladičem
(15/20\gc), vodním chladičem (9/15\gc), vodním ohřívačem (45/30\gc) a dvěma
ventilátory s frekvenčními měniči. 

Zařízení pracuje s nuceným přívodem a odvodem
vzduchu. Čerstvý vzduch je nasáván VZT jednotkou potrubím z fasády objektu, kde
je umístěna protidešťová žaluzie. Znehodnocený vzduch je VZT jednotkou vyfukován
nad střechu objektu.       

}{

}

\twoColls{
Návrhové parametry jednotky, jsou uvedeny v příloze: TABLE EQUIPMENT AND
PRINCIPLE DIAGRAMS.

}{
Unit design parameters are listed in annex: TABLE EQUIPMENT AND PRINCIPLE
DIAGRAMS.

}

\twoColls{
Hygienické místnosti, jsou větrány podtlakovým způsobem pomocí samostatného
odsávacího zařízení složeného z potrubí, odsávacího ventilátoru umístěného ve
strojovně AC v 2.NP. a přetlakové klapky. Znehodnocený vzduch je z jednotlivých
místností odsáván pomocí vyústek osazených do podhledu a je ventilátorem
vyfukován na střechu objektu. Náhradní vzduch je přisáván přes stěnové mřížky a
pod dveřmi z prostoru chodby.

}{

}

\twoColls{
Návrhové parametry ventilátoru, jsou uvedeny v příloze: TABLE EQUIPMENT AND
PRINCIPLE DIAGRAMS.

}{
Fan design parameters are listed in annex: TABLE EQUIPMENT AND PRINCIPLE
DIAGRAMS.

}

\twoColls{
Sklad odpadků, je větrán podtlakovým způsobem pomocí samostatného odsávacího
zařízení složeného z potrubí, odsávacího ventilátoru umístěného ve strojovně AC
v 2.NP. a přetlakové klapky. Znehodnocený vzduch je z místnosti odsáván pomocí
vyústky osazené do podhledu, a je ventilátorem vyfukován na střechu objektu.
Náhradní vzduch je přisáván přes požární klapku z prostoru chodby. 

}{

}

\twoColls{
Návrhové parametry ventilátoru, jsou uvedeny v příloze: TABLE EQUIPMENT AND
PRINCIPLE DIAGRAMS.

}{
Fan design parameters are listed in annex: TABLE EQUIPMENT AND PRINCIPLE
DIAGRAMS.

}

\twoColls{
Větrání cooling machine room s minimální tepelnou zátěží od technologie
(kondenzátory od technologie chlazení jsou umístěny na střeše), je řešeno
systémem podtlakového odsávání pomocí potrubního ventilátoru. Ventilátor se
spouští na základě prostorového teplotního čidla a při vstupu do místnosti.  

}{

}

\twoColls{
Venkovní vzduchu je přiváděn ze společného nasávání ve strojovně v 2.NP. Přívod
zajišťuje potrubního ventilátor přes klapku a filtr G4. Vzduchový výkon je
navržen na 10x výměn vzduchu za hodinu.

}{

}

\twoColls{
Návrhové parametry ventilátoru, jsou uvedeny v příloze: TABLE EQUIPMENT AND
PRINCIPLE DIAGRAMS.

}{
Fan design parameters are listed in annex: TABLE EQUIPMENT AND PRINCIPLE
DIAGRAMS.

}


\clearpage
\section{Equipment No.09.05-4	- Breaking Area}

\clearpage
\section{Equipment No.09.05-5	- Doctor}

\clearpage
\section{Equipment No.09.05-6	- Sanitary room}

\clearpage
\section{Equipment No.09.05-7	- AC machinery room}

\clearpage
\section{Equipment No.09.05-8	- Switch rooms}

\clearpage
\section{Equipment No.09.05-9	- Protected escape route}

\twoColls{

}{

}
