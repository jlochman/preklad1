\clearpage
\section{Equipment No.09.05-1	- Office area}

\twoColls{
  Větrání prostor kanceláří a jednacích místnosti zajišťuje jedna vzduchotechnická
  jednotka osazená ve strojovně vzduchotechniky v 2NP. Teplotně upravovaný
  venkovní vzduch je přiváděn potrubím do vnitřních prostor pomocí vířivých
  anemostatů, jež jsou osazeny do podhledu, a znehodnocený vzduch je odsáván
  potrubím vedeným nad podhledem. Propojení mezi větranou místností a odsávaným
  podhledem, bude zajištěno mřížovanou podhledovou kazetou. 

  Pro každou větranou
  zónu (kancelář, zasedací místnost, archiv, čajová kuchyňka), budou do přívodního
  potrubí osazeny regulátory průtoku VAV s vodními ohřívači a do odvodního potrubí
  regulátory průtoku VAV. Regulátory průtoku VAV s ohřívači na přívodu budou
  regulovány dle prostorového teplotního čidla z každé místnosti. Regulátory VAV
  na odvodu budou regulovány souběžně s regulátory VAV na přívodu. Otáčky
  ventilátorů v centrálních jednotkách budou řízeny na konstantní tlak v potrubí.

}{
  Office areas and meeting rooms will be ventilated by single AHU which will be
  placed in AC machinery room on the 2nd floor.  Thermally conditioned outdoor air
  will be supplied through ventilation duct to indoor areas by swirl diffusers,
  which will be installed in ceiling.  Waste air will be exhausted by ducts
  leading above the ceiling.  Ventilated office will be interconnected with
  exhausting ceiling by reticulated cartridges mounted in the ceiling. 

  For each ventilated area (office, meeting room, management archive or tea
  kitchen), supply ducts will be equipped with flow rate VAV controllers with
  water-based air heaters, whereas the exhaust ducts will be equipped with flow
  rate VAV controllers only. Whole equipment in inlets will be controlled
  according to temperature sensor in each room. As soon as these controllers are
  set, controller settings on outlets will be determined.  Fan speeds in central
  AHUs will be set to ensure constant pressure in supply and exhaust ducts.

}

\twoColls{
  V každé místnosti bude umístěno čidlo kvality vzduchu, a poměr směšování
  v centrální VZT jednotce, bude řízen dle nejhorší hodnoty ze všech čidel.
  Přiváděná dávka venkovního vzduchu do kanceláře je 50m3/h na osobu a do zasedací
  místnosti 80m3/h na osobu, nebo min. 4x/h.

}{
  Air quality sensor will be placed in each room and according to the worse
  sensor value, the mixing ration of fresh air will be set in central AHU. The
  amount of outdoor air supplied to offices is calculated for 50~m3/h per person
  and for meeting rooms for 80~m3/h per person. There will always be at least 4
  air exchanges per hour.

}

\twoColls{
  Vzduchotechnická jednotka je ve vnitřním provedení, opatřena filtrem vzduchu
  (přívod F5+F7 /odtah F5), rotačním výměníkem, směšovací komorou, vodním
  chladičem (15/20\gc), vodním chladičem (9/15\gc), vodním ohřívačem (45/30\gc) a
  dvěma ventilátory s frekvenčními měniči. 

  Zařízení pracuje s nuceným přívodem a
  odvodem vzduchu. Čerstvý vzduch je nasáván VZT jednotkou potrubím z fasády
  objektu, kde je umístěna protidešťová žaluzie. Znehodnocený vzduch je VZT
  jednotkou vyfukován nad střechu objektu.       

}{
  Innerly designed AHU will be equipped with air filters (F5+F7 supply / F5
  exhaust), rotatory heat exchanger, mixing chamber, water cooler (15 / 20\gc),
  water cooler (9 / 15\gc), water heater (45 / 30\gc) and two fans with frequency
  changers.

  Equipment will work with forced air inlet and outlet. Fresh air will be
  suctioned in by AHU through ducts from building facade where a weather
  resistant louver will be placed. Waste air will be blown by AHU over the roof
  of the building.

}

\twoColls{
  Chlazení kanceláří je zajištěno pomocí centrální VZT jednotky (VAV systém).
  Maximální přiváděné množství vzduchu do kanceláře a zasedací místnosti je
  navrženo na odvod letní tepelné zátěže, snížení množství přívodního vzduchu
  zajistí minimální hygienické větrání. V zasedací místnosti bude zajištěna
  možnost zcela uzavřít přívod a odvod vzduchu (zajistí profese MaR) tlačítkem
  z prostoru místnosti. 

}{
  Offices cooling will be ensured by central AHU (variable air volume system –
  VAV). Maximal amount of air supplied to offices and meeting room is designed to
  remove heat gains in summer time. Amount of supplied air will always comply with
  a requirement of minimal hygienic ventilation.  In meeting room, there is a
  possibility of complete closure of the ventilation systems by the switches
  (ensured by profession M\&C – Measurement and Control). 

}

\twoColls{
  Návrhové parametry jednotky, jsou uvedeny v příloze: TABLE EQUIPMENT AND
  PRINCIPLE DIAGRAMS.

}{
  Unit design parameters are listed in annex: TABLE EQUIPMENT AND PRINCIPLE
  DIAGRAMS.

}

\clearpage
\section{Equipment No.09.05-2	- Caffeteria}

\twoColls{
  Větrání, chlazení a vytápění jídelny zajišťuje jedna vzduchotechnická jednotka
  osazená ve strojovně vzduchotechniky v 2NP. Teplotně upravovaný venkovní vzduch
  je přiváděn potrubím do prostoru, kam je vyfukován pomocí vířivých vyústek. 

}{
  Ventilation, cooling and heating of Caffeteria will be ensured by single AHU
  placed in AC machinery room on the 2nd floor. Thermally conditioned outdoor air
  will be forced through ventilation duct to caffeteria, where it will be
  blown out through swirl diffusers. 

}

\twoColls{
  Odvod vzduchu je pomocí dvou potrubních tras vedených nad podhledem.  Jedno
  odsávací potrubí je místně vyústěno nad podhledem a přisávání vzduchu je
  zajištěno přes mřížované podhledové kazety.  Druhé odsávací potrubí je vedeno
  nad výdejem jídel a do potrubí jsou napojeny odsávací vyústky-odlučovače tuku
  osazené do podhledu. 

}{
  Air exhaust will be ensured by two duct lines leading above the suspended
  ceiling. First of these lines will locally terminate above the suspended
  ceiling and will suction air in through perforated ceiling grids. Second
  line will be led above food serving and to its duct, outlets equipped with
  grease separating diffusers will be connected in suspended ceiling.

}

\twoColls{
  Množství přiváděného vzduchu je navrženo na 10 výměn vzduchu za hodinu, tak aby
  byla odvedena tepelná zátěž z prostru. Při přiváděném množství vzduchu
  13.500m3/h a počtu 180 osob, potom vychází dávka venkovního vzduchu
  75m3/h/osobu.

}{
  Amount of supplied air is designed to ensure 10 air exchanges per hour in
  order to lead heat load away. Assuming 13,500~m3/h of supply air and
  180~persons, there will be 75~m3/h of supplied air per person.

}

\twoColls{
  Vzduchotechnická jednotka je ve vnitřním provedení, opatřena filtrem vzduchu
  (přívod F5+F7 /odtah GREASE FILTER+F5), deskovým výměníkem s obtokem, vodním
  chladičem (15/20\gc), vodním chladičem (9/15\gc), vodním ohřívačem (45/30\gc) a
  dvěma ventilátory s frekvenčními měniči. 

  Zařízení pracuje s nuceným přívodem a
  odvodem vzduchu. Čerstvý vzduch je nasáván VZT jednotkou potrubím z fasády
  objektu, kde je umístěna protidešťová žaluzie. Znehodnocený vzduch je VZT
  jednotkou vyfukován nad střechu objektu.       

}{
  Innerly designed AHU will be equipped with air filters (F5+F7 supply /
  F5+GREASE FILTER exhaust), plate heat exchanger with bypass, water-based air
  cooler (15 / 20\gc), water-based air cooler (9 / 15\gc) water-based air
  heater (45 / 30\gc) and two fans with frequency changers. 

  Equipment will work with forced air inlet and outlet. Fresh air will be
  suctioned in by AHU through ducts from building facade, where a weather
  resistant louver will be placed.  Waste air will be blown by AHU over the roof
  of the building.

}

\twoColls{
  Návrhové parametry jednotky, jsou uvedeny v příloze: TABLE EQUIPMENT AND
  PRINCIPLE DIAGRAMS.

}{
  Unit design parameters are listed in annex: TABLE EQUIPMENT AND PRINCIPLE
  DIAGRAMS.

}


\clearpage
\section{Equipment No.09.05-3	- Kitchen}

\twoColls{
  Větrání, chlazení a vytápění prostor kuchyně zajišťují dvě vzduchotechnické
  jednotky osazené ve strojovně vzduchotechniky v 2NP. Jednotka poz. 3.1 zajišťuje
  větrání varny, studené kuchyně (cold kitchen) a dvou umýváren nádobí. Jednotka
  poz. 3.2 zajišťuje větrání menších místností náležících k zázemí kuchyně (šatny,
  kanceláře, chodby, sklady atd.).

}{
  Ventilation, cooling and heating of Kitchen will be ensured by two AHUs placed
  in AC machinery room on the 2nd floor. Unit ref. 3.1 will ensure ventilation
  of brew room, cold kitchen and dish washrooms. Unit ref. 3.2 will ensure
  ventilation of smaller rooms belonging to Kitchen facilities (dressing rooms,
  offices, corridors, warehouses etc.)

}

\twoColls{
  Jednotkou poz. 3.1 jsou větrány místnosti KITCHEN, COLD KITCHEN a 2x WASHING. 

}{
  Unit ref. 3.1 will ensure ventilation of KITCHEN, COLD KITCHEN and 2x WASHING.

}

\twoColls{
  Teplotně upravovaný venkovní vzduch je přiváděn potrubím do vnitřních prostor
  kuchyně z větší části pomocí velkoplošných vyústek integrovaných do odsávacích
  zákrytů a z menší části pomocí vířivých anemostatů, jež jsou osazeny do
  podhledu. Do studené kuchyně a umýváren je upravený vzduch přiváděn pomocí
  vířivých anemostatů, jež jsou osazeny do podhledu.

}{
  Thermally conditioned outdoor air will be supplied to indoor areas of kitchen
  through ventilation duct equipped mostly with large-area inlets integrated
  into the extraction covers and, in some cases, with swirl diffusers, which
  will be installed in the suspended ceiling.  Air will be supplied to cold
  kitchen and washing rooms through swirl diffusers installed in the suspended
  ceiling.

}

\twoColls{
  Znehodnocený vzduch je odsáván z kuchyně a mytí nádobí z větší části pomocí
  odsávacích zákrytů a z menší části pomocí vyústek-odlučovačů tuků osazených do
  podhledů. 

}{
  Waste air will be exhausted from kitchen and washing rooms mostly through
  outlet covers and in some cases through grease separating outlets installed in
  suspended ceiling as well.

}

\twoColls{
  V kuchyni budou ve středu místnosti nad varnými spotřebiči umístěny nerezové
  odsávací digestoře HALTON s integrovanými velkoplošnými vyústkami pro přívod
  vzduchu. Obě velké digestoře budou vybaveny hasícím systémem ANSUL. Boxy ANSUL
  budou osazeny na boku digestoře. Nad konvektomatem bude umístěna nerezová
  odsávací digestoř HALTON s integrovanými velkoplošnými vyústkami pro přívod
  vzduchu. Nad myčkami v místnostech mytí nádobí, budou umístěny kondenzační
  nerezové odsávací digestoře HALTON.

}{
  In the center of kitchen, above the cooking appliances, there will be
  installed two HALTON digesters with integrated large-area air-supplying
  inlets.  Both digesters will be equipped with fire extinguishing system ANSUL.
  ANSOL boxes will be installed on one of the digester sides.  Above convection
  oven, stainless steel HALTON digester with integrated large-area air-supplying
  inlets will be installed.  Above dishwashers in washing rooms, condensing
  stainless steel exhaust HALTON digesters will be installed.

}

\twoColls{
  Vzduchotechnická jednotka je ve vnitřním provedení, opatřena filtrem vzduchu
  (přívod F5+F7 /odtah GREASE FILTER+F5), deskovým výměníkem s obtokem, vodním
  chladičem (15/20\gc), vodním chladičem (9/15\gc), vodním ohřívačem (45/30\gc) a
  dvěma ventilátory s frekvenčními měniči. 

  Zařízení pracuje s nuceným přívodem a odvodem vzduchu. Čerstvý vzduch je nasáván
  VZT jednotkou potrubím z fasády objektu, kde je umístěna protidešťová žaluzie.
  Znehodnocený vzduch je VZT jednotkou vyfukován nad střechu objektu.       

}{
  Innerly designed AHU will be equipped with air filters (F5+F7 supply /
  F5+GREASE FILTER exhaust), plate heat exchanger with bypass, water-based air
  cooler (15 / 20\gc), water-based air cooler (9 / 15\gc) water-based air
  heater (45 / 30\gc) and two fans with frequency changers. 

  Equipment will work with forced air inlet and outlet. Fresh air will be
  suctioned in by AHU through ducts from building facade, where a weather
  resistant louver will be placed.  Waste air will be blown by AHU over the roof
  of the building.

}

\twoColls{
  Návrhové parametry jednotky, jsou uvedeny v příloze: TABLE EQUIPMENT AND
  PRINCIPLE DIAGRAMS.

}{
  Unit design parameters are listed in annex: TABLE EQUIPMENT AND PRINCIPLE
  DIAGRAMS.

}

\twoColls{
  Jednotkou poz. 3.2, jsou větrány místnosti OFFICE, CORRIDOR, CLOAK ROOM SHOWER,
  STORAGE, WASHING THERMOPORTS.   

}{
  Unit ref. 3.2 will ensure ventilation of OFFICE, CORRIDOR, CLOAK ROOM SHOWER,
  STORAGE, WASHING THERMOPORTS.   

}

\twoColls{
  Teplotně upravovaný venkovní vzduch je přiváděn potrubím do vnitřních prostor
  pomocí vířivých anemostatů, jež jsou osazeny do podhledu. Znehodnocený vzduch je
  odsáván potrubím vedeným nad podhledem. Propojení mezi větranou místností a
  odsávaným podhledem, bude zajištěno mřížovanou podhledovou kazetou. 

}{
  Thermally conditioned outdoor air will be forced through ventilation duct to
  indoor areas, where it will be blown out through swirl diffusers mounted in
  suspended ceiling.  Waste air will be exhausted through
  an exhaust duct below the ceiling. Ventilated room will be
  interconnected with ventilation ducts by reticulated cartridges.

}

\twoColls{
  Jelikož, je jedním centrálním zařízením klimatizováno více prostor, jsou
  v přívodním potrubí do kanceláří a šaten osazeny regulátory průtoku VAV
  s vodními ohřívači a v odvodním potrubí regulátory průtoku VAV. Regulátory
  průtoku VAV s ohřívači na přívodu budou regulovány dle prostorového teplotního
  čidla z každé místnosti. 

  Pro ostatní prostory jsou skupinově do přívodního a
  odsávacího potrubí osazeny pouze regulátory VAV. Regulátory VAV na odvodu budou
  regulovány souběžně s regulátory VAV na přívodu. Otáčky ventilátorů
  v centrálních jednotkách budou řízeny na konstantní tlak v potrubí. Množství
  přiváděného vzduchu je navrženo dle standardu LEGO.  

}{
  Since many areas will be climatized by one central equipment, cloak room and
  office inlets will be equipped with flow rate VAV controllers with water-based
  air heaters whereas outlets will be equipped with flow rate VAV controllers
  only. Whole equipment in inlets will be controlled according to temperature
  room sensor.  

  Inlets and outlets in remaining areas will be equipped with flow rate VAV
  controllers only. Outlet VAV controllers will be controlled in parallel with
  controllers in inlets. Fan speeds in central units will be controlled
  to ensure constant pressure in ducts. The amount of supplied air is designed
  to comply with LEGO standard.

}

\twoColls{
  Vzduchotechnická jednotka je ve vnitřním provedení, opatřena filtrem vzduchu
  (přívod F5+F7 /odtah F5), deskovým výměníkem s obtokem, vodním chladičem
  (15/20\gc), vodním chladičem (9/15\gc), vodním ohřívačem (45/30\gc) a dvěma
  ventilátory s frekvenčními měniči. 

  Zařízení pracuje s nuceným přívodem a odvodem
  vzduchu. Čerstvý vzduch je nasáván VZT jednotkou potrubím z fasády objektu, kde
  je umístěna protidešťová žaluzie. Znehodnocený vzduch je VZT jednotkou vyfukován
  nad střechu objektu.       

}{
  Innerly designed AHU will be equipped with air filters (F5+F7 supply / F5
  exhaust), plate heat exchanger with bypass, water-based air cooler (15 /
  20\gc), water-based air cooler (9 / 15\gc) water-based air heater (45 / 30\gc)
  and two fans with frequency changers. 

  Equipment will work with forced air inlet and outlet. Fresh air will be
  suctioned in by AHU through ducts from building facade, where a weather
  resistant louver will be placed.  Waste air will be blown by AHU over the roof
  of the building.

}

\twoColls{
  Návrhové parametry jednotky, jsou uvedeny v příloze: TABLE EQUIPMENT AND
  PRINCIPLE DIAGRAMS.

}{
  Unit design parameters are listed in annex: TABLE EQUIPMENT AND PRINCIPLE
  DIAGRAMS.

}

\twoColls{
  Hygienické místnosti, jsou větrány podtlakovým způsobem pomocí samostatného
  odsávacího zařízení složeného z potrubí, odsávacího ventilátoru umístěného ve
  strojovně AC v 2.NP. a přetlakové klapky. Znehodnocený vzduch je z jednotlivých
  místností odsáván pomocí vyústek osazených do podhledu a je ventilátorem
  vyfukován na střechu objektu. Náhradní vzduch je přisáván přes stěnové mřížky a
  pod dveřmi z prostoru chodby.

}{
  Sanitary rooms will be ventilated by an underpressure way by an usage of
  separate suction device composed of ducts, exhaust fan installed in AC
  machinery room on the 2nd floor and overpressure valve. Waste air will be
  exhausted from individual rooms with diffusers mounted in the ceiling and
  blown by ventilator over the roof of the building. Replacement air will be
  suctioned through wall grilles and from corridor through space under the door.

}

\twoColls{
  Návrhové parametry ventilátoru, jsou uvedeny v příloze: TABLE EQUIPMENT AND
  PRINCIPLE DIAGRAMS.

}{
  Fan design parameters are listed in annex: TABLE EQUIPMENT AND PRINCIPLE
  DIAGRAMS.

}

\twoColls{
  Sklad odpadků, je větrán podtlakovým způsobem pomocí samostatného odsávacího
  zařízení složeného z potrubí, odsávacího ventilátoru umístěného ve strojovně AC
  v 2.NP. a přetlakové klapky. Znehodnocený vzduch je z místnosti odsáván pomocí
  vyústky osazené do podhledu, a je ventilátorem vyfukován na střechu objektu.
  Náhradní vzduch je přisáván přes požární klapku z prostoru chodby. 

}{
  Waste storage will be ventilated by an underpressure way by an usage of
  separate suction device composed of ducts, exhaust fan installed in AC
  machinery room on the 2nd floor and overpressure valve. Waste air will be
  exhausted from individual rooms with diffusers mounted in the ceiling and
  blown by ventilator over the roof of the building. Replacement air will be
  suctioned through fire damper from corridor space.

}

\twoColls{
  Návrhové parametry ventilátoru, jsou uvedeny v příloze: TABLE EQUIPMENT AND
  PRINCIPLE DIAGRAMS.

}{
  Fan design parameters are listed in annex: TABLE EQUIPMENT AND PRINCIPLE
  DIAGRAMS.

}

\twoColls{
  Větrání cooling machine room s minimální tepelnou zátěží od technologie
  (kondenzátory od technologie chlazení jsou umístěny na střeše), je řešeno
  systémem podtlakového odsávání pomocí potrubního ventilátoru. Ventilátor se
  spouští na základě prostorového teplotního čidla a při vstupu do místnosti.  

}{
  Ventilation of cooling machine room having minimal heat load from technology
  (condensers of cooling technology will be installed on the roof) will be ensured
  by a system of vacuum suction by fan in duct. Fan will be triggered based on
  room temperature sensor and when a person enters the room.

}

\twoColls{
  Venkovní vzduchu je přiváděn ze společného nasávání ve strojovně v 2.NP. Přívod
  zajišťuje potrubního ventilátor přes klapku a filtr G4. Vzduchový výkon je
  navržen na 10x výměn vzduchu za hodinu.

}{
  Outdoor air will be suctioned in from shared suction in AC machinery room on
  the 2nd floor. Supply will be ensured by a duct fan through valve and G4 filter.
  The amount of supplied air is designed to ensure 10 air exchanges per hour.

}

\twoColls{
  Návrhové parametry ventilátoru, jsou uvedeny v příloze: TABLE EQUIPMENT AND
  PRINCIPLE DIAGRAMS.

}{
  Fan design parameters are listed in annex: TABLE EQUIPMENT AND PRINCIPLE
  DIAGRAMS.

}


\clearpage
\section{Equipment No.09.05-4	- Breaking Area}

\twoColls{
  Větrání breaiking area a technical archive, zajišťuje jedna vzduchotechnická
  jednotka osazená ve strojovně vzduchotechniky v 2NP. Teplotně upravovaný
  venkovní vzduch je přiváděn potrubím do vnitřních prostor pomocí vířivých
  anemostatů, jež jsou osazeny do podhledu, a znehodnocený vzduch je odsáván
  potrubím vedeným nad podhledem. Propojení mezi větranou místností a odsávaným
  podhledem, bude zajištěno mřížovanou podhledovou kazetou. 

}{
  Breaking area and technical archive will be ventilated by single AHU placed in AC
  machinery room on the 2nd floor. Thermally conditioned outdoor air will be supplied
  through ventilation duct to indoor areas by swirl diffusers mounted in
  suspended ceiling. Waste air will be exhausted by ducts leading over the suspended
  ceiling. Connection between ventilated room and suctioned suspended ceiling
  will be ensured by perforated ceiling grids.

}

\twoColls{
  Tepelný výkon jednotky bude řízen dle prostorového čidla teploty v breaking
  area.  Jelikož, je jedním centrálním zařízením větráno více prostor, jsou
  v přívodním i odvodním potrubí osazeny regulátory průtoku VAV. Regulátory
  průtoku VAV  na přívodu budou regulovány dle prostorového teplotního čidla
  z každé místnosti. Regulátory VAV na odvodu budou regulovány souběžně
  s regulátory VAV na přívodu. Otáčky ventilátorů v centrální jednotce budou
  řízeny na konstantní tlak v potrubí. 

}{
  AHU heat power will be controlled by a room temperature sensor placed in the
  breaking area.  Since many areas will be ventilated by one central equipment,
  inlets and outlets will be equipped with flow rate VAV controllers. VAV
  controllers in inlets will be set according to temperature sensor in each
  room. As soon as these controllers are set, controller settings on outlets are
  determined.  Fan speeds in central AHUs will be set to ensure constant
  pressure in supply and exhaust ducts.

}

\twoColls{
  Poměr směšování vzduchu v centrální VZT jednotce, bude řízen dle čidla kvality
  vzduchu umístěného v breaking room. Přiváděná dávka venkovního vzduchu do
  breaking room je 50m3/h na osobu, nebo min. 4x/h.

}{
  The air mixing ratio in central AHU will be set according to air quality
  sensor in breaking room. The amount of outdoor air supplied to offices is
  calculated fo 50~m3/h per person. There will always be at least 4 air
  exchanges per hour.

}

\twoColls{
  Vzduchotechnická jednotka je ve vnitřním provedení, opatřena filtrem vzduchu
  (přívod F5+F7 /odtah F5), rotačním výměníkem, směšovací komorou, vodním
  chladičem (15/20\gc), vodním ohřívačem (45 / 30\gc) a dvěma ventilátory
  s frekvenčními měniči. 

  Zařízení pracuje s nuceným přívodem a odvodem vzduchu.
  Čerstvý vzduch je nasáván VZT jednotkou potrubím z fasády objektu, kde je
  umístěna protidešťová žaluzie. Znehodnocený vzduch je VZT jednotkou vyfukován
  nad střechu objektu.       

}{
  Innerly designed AHU will be equipped with air filters (F5+F7 supply / F5
  exhaust), rotatory heat exchanger, mixing chamber, water-based air cooler (15
  / 20\gc), water-based air heater (45 / 30\gc) and two fans with frequency
  changers. 

  Equipment will work with forced air inlet and outlet. Fresh air will be
  suctioned in by two AHUs through ducts from building facade, where a weather
  resistant louver will be placed.  Waste air will be blown by AHU over the roof
  of the building.

}

\twoColls{
  Návrhové parametry jednotky, jsou uvedeny v příloze: TABLE EQUIPMENT AND
  PRINCIPLE DIAGRAMS.

}{
  Unit design parameters are listed in annex: TABLE EQUIPMENT AND PRINCIPLE
  DIAGRAMS.

}

\clearpage
\section{Equipment No.09.05-5	- Doctor}

\twoColls{
  Větrání místností doctor, waiting room, first aid zajišťuje jedna
  vzduchotechnická jednotka osazená ve strojovně vzduchotechniky v 2NP. Teplotně
  upravovaný venkovní vzduch je přiváděn potrubím do vnitřních prostor pomocí
  vířivých anemostatů, jež jsou osazeny do podhledu, a znehodnocený vzduch je
  odsáván potrubím vedeným nad podhledem. Propojení mezi větranou místností a
  odsávaným podhledem, bude zajištěno mřížovanou podhledovou kazetou. 

  Pro každou
  větranou zónu, budou do přívodního potrubí osazeny regulátory průtoku VAV
  s vodními ohřívači a do odvodního potrubí regulátory průtoku VAV. Regulátory
  průtoku VAV s ohřívači na přívodu budou regulovány dle prostorového teplotního
  čidla z každé místnosti. Regulátory VAV na odvodu budou regulovány souběžně
  s regulátory VAV na přívodu. Otáčky ventilátorů v centrálních jednotkách budou
  řízeny na konstantní tlak v potrubí.  

}{
  Doctor, waiting and first aid rooms will be ventilated by single AHU which
  will be placed in AC machinery room on the 2nd floor. Thermally conditioned
  outdoor air will be supplied through ventilation duct to indoor areas by swirl
  diffusers, which will be installed in ceiling. Waste air will be exhausted by
  ducts leading above the ceiling. Ventilated rooms will be interconnected with
  exhausting ceiling by reticulated cartridges mounted in the ceiling. 

  For each ventilated area, supply ducts will be equipped with flow rate VAV
  controllers with water-based air heaters, whereas the exhaust ducts will be
  equipped with flow rate VAV controllers only. Whole equipment in inlets will
  be controlled according to temperature sensor in each room. As soon as these
  controllers are set, controller settings on outlets will be determined.  Fan
  speeds in central AHUs will be set to ensure constant pressure in supply and
  exhaust ducts.

}

\twoColls{
  Přiváděná dávka venkovního vzduchu je min. 50m3/h na osobu, nebo min. 4x/h.

}{
  The amount of supplied outdoor air is calculated fo 50~m3/h per person. There
  will always be at least 4 air exchanges per hour.

}

\twoColls{
  Vzduchotechnická jednotka je ve vnitřním provedení, opatřena filtrem vzduchu
  (přívod F5+F7 /odtah F5), deskovým výměníkem s obtokem, vodním chladičem
  (9/15\gc), vodním ohřívačem (45/30\gc) a dvěma ventilátory s frekvenčními
  měniči. 

  Zařízení pracuje s nuceným přívodem a odvodem vzduchu. Čerstvý vzduch
  je nasáván VZT jednotkou potrubím z fasády objektu, kde je umístěna
  protidešťová žaluzie. Znehodnocený vzduch je VZT jednotkou vyfukován nad
  střechu objektu.       

}{
  Innerly designed AHU will be equipped with air filters (F5+F7 supply /
  F5 exhaust), plate heat exchanger with bypass, water-based air
  cooler (9 / 15\gc), water-based air
  heater (45 / 30\gc) and two fans with frequency changers. 

  Equipment will work with forced air inlet and outlet. Fresh air will be
  suctioned in by AHU through ducts from building facade, where a weather
  resistant louver will be placed.  Waste air will be blown by AHU over the roof
  of the building.

}

\twoColls{
  Návrhové parametry jednotky, jsou uvedeny v příloze: TABLE EQUIPMENT AND
  PRINCIPLE DIAGRAMS.

}{
  Unit design parameters are listed in annex: TABLE EQUIPMENT AND PRINCIPLE
  DIAGRAMS.

}

\clearpage
\section{Equipment No.09.05-6	- Sanitary room}

\twoColls{
  Hygienické místnosti, jsou větrány podtlakovým způsobem pomocí samostatných
  odsávacích zařízení složených z potrubí, odsávacího ventilátoru umístěného pod
  stropem větrané místnosti a uzavírací klapky. Znehodnocený vzduch je
  z jednotlivých místností odsáván pomocí vyústek osazených do podhledu a je
  ventilátorem vyfukován na střechu objektu. Náhradní vzduch je přisáván přes
  stěnové mřížky a pod dveřmi z prostoru chodby.

}{
  Sanitary rooms will be ventilated be a vacuum way using a separate exhaust
  system consisting of ducts, exhaust fan mounted below the ceiling of
  ventilated area and overpressure valve. Waste air from individual sanitary
  rooms will be exhausted through ventilation valves mounted in the suspended
  ceiling and blown over the roof by fan.  Replacement air will be supplied
  through wall grilles and through the space under the door from corridor area. 

}

\twoColls{
  Návrhové parametry ventilátoru, jsou uvedeny v příloze: TABLE EQUIPMENT AND
  PRINCIPLE DIAGRAMS.

}{
  Fan design parameters are listed in annex: TABLE EQUIPMENT AND PRINCIPLE
  DIAGRAMS.

}

\clearpage
\section{Equipment No.09.05-7	- AC machinery room}

\twoColls{
  Větrání strojovny vzduchotechniky v a 2.NP je řešeno pomocí 3ks střešních
  ventilátorů. Vzduchový výkon je navržen na 3x výměnu vzduchu za hodinu.
  Ventilátory budou osazeny na hluk tlumícím soklu a budou vybaveny zpětnou
  klapkou. Náhradní vzduch bude přisáván z venkovního prostoru přes 6ks sestav
  složených z protidešťové žaluzie, uzavírací klapky a filtru G4.

}{
  Ventilation of AC machinery room on the 2nd floor will be ensured by 3 pcs of
  roof fan. Air flow rate is designed for 3 air exchanges per hour. Fans will be
  mounted on noise-absorbing base and equipped with back-draft damper.
  Replacement air will be supplied through 6 pcs of assembly consisting of
  weather resistant louver, shut-off damper and G4 filter.

}

\twoColls{
  Návrhové parametry ventilátoru, jsou uvedeny v příloze: TABLE EQUIPMENT AND
  PRINCIPLE DIAGRAMS.

}{
  Fan design parameters are listed in annex: TABLE EQUIPMENT AND PRINCIPLE
  DIAGRAMS.

}

\twoColls{
  Odvod tepelné zátěže z cooling equipment, bude zajištěn kombinovaným systémem
  nuceného podtlakového odsávání a chlazení vzduchu pomocí cirkulační jednotky
  AHU. V chladném období roku, bude místnost chlazena pouze venkovním vzduchem.
  Venkovní vzduch, bude do místnosti nasáván z fasády přes filtr G4. V teplém
  období roku, bude odsávací střešní ventilator vypnut a tepelná zátěž z
  místnosti bude odváděna pomocí cirkulační jednotky AHU s vodním chladičem.

}{
  Exhaust of heat load from cooling equipment will be ensured by a combined
  system of forced vacuum suction and air cooling by circulation AHU. During a cold
  year season, machinery room will be cooled by outdoor air only, which will be
  suctioned into machinery room through a G4 filter from building facade.
  During a warm year season, an exhausted ventilator will be turned off and a
  heat load from machinery room will be exhausted by the circulation AHU with a
  water-based air cooler only.


}

\twoColls{
  Vzduchotechnická jednotka je ve vnitřním provedení, opatřena filtrem vzduchu
  G4, vodním chladičem (15/20\gc) a ventilátorem s frekvenčním měničem. 

}{
  Innerly designed AHU will be equipped with G4 air filter, water-based air
  cooler (15 / 20\gc) and fan with frequency changer.

}

\twoColls{
  Návrhové parametry jednotky a ventilátoru, jsou uvedeny v příloze: TABLE
  EQUIPMENT AND PRINCIPLE DIAGRAMS.

}{
  Fan and unit design parameters are listed in annex: TABLE EQUIPMENT AND
  PRINCIPLE DIAGRAMS.

}

\clearpage
\section{Equipment No.09.05-8	- Switch rooms}

\twoColls{
  Větrání technických místností s minimální tepelnou zátěží od technologie, je
  řešeno systémem podtlakového odsávání pomocí potrubního ventilátoru.
  Ventilátor se spouští na základě prostorového teplotního čidla a při vstupu do
  místnosti.  

}{
  Ventilation of technical rooms having minimal heat load from technology will be
  ensured by a system of vacuum suction by duct fan. Fan will be triggered
  based on room temperature sensor and when a person enters the room.

}

\twoColls{
  Přívod venkovního vzduchu je řešen pomocí potrubních ventilátorů s uzvíracími
  klapkami a společného filtru G4 a protidešťové  žaluzie. 

}{
  Outdoor air are will be supplied by duct fans through weather resistant louver,
  shut-off damper and G4 filter. 

}

\twoColls{
  Vzduchový výkon v místnosti switch room je navržen pro chladné období roku na
  tepelnou zátěž.  Chlazení místnosti switch room na předepsanou teplotu
  zajišťuje profese CHLAZENÍ.

}{
  The amount of air supplied to switch room is designed to exhaust a heat load
  during a cold year season. Cooling of switch room at a prescribed temperature
  is ensured by profession of COOLING.

}

\twoColls{
  Vzduchový výkon v místnosti heating equipment je navržen na 3x výměnu vzduchu
  za hodinu.

}{
  The amount of air supplied to heating equipment room is designed to ensure 3
  air exchange per hour

}

\twoColls{
  Návrhové parametry ventilátoru, jsou uvedeny v příloze: TABLE EQUIPMENT AND
  PRINCIPLE DIAGRAMS.

}{
  Fan design parameters are listed in annex: TABLE EQUIPMENT AND PRINCIPLE
  DIAGRAMS.

}

\clearpage
\section{Equipment No.09.05-9	- Protected escape route}

\twoColls{
  Větrání chráněné únikových cest (CHÚC) je řešeno nuceným přetlakovým způsobem.
  Venkovní vzduch je nasáván potrubním ventilátorem nad střechou objektu a pomocí
  potrubí, je přiveden do prostoru CHÚC. Přívodní potrubí vedené jiným požárním
  úsekem, bude požárně izolováno.

}{
  Protected escape route (PER) will be ventilated by a forced overpressure way.
  Outdoor air will be supplied by a duct ventilator above the roof of the
  building and led through a duct to the areas of PER. Supply duct leading
  through fire compartments will be provided with a fire insulation. 

}

\twoColls{
  Odvod vzduchu z větraného prostoru je zajištěn vlivem přetlaku v nejvyšším místě
  CHÚC. V odvodním potrubí bude osazena uzavírací a přetlaková klapka, na které
  bude nastaven požadovaný přetlak cca 25Pa. Celkem budou použity 2ks výfukových
  kompletů. Přiváděné množství vzduchu do CHÚC zajistí 10 výměn vzduchu za hodinu.

}{
  Air exhaust from ventilated area will be ensured by overpressure in the
  highest point of PER. The outlet will be provided with an overpressure valve
  on its inner side, which will be set to ensure the overpressure of ca. 25~Pa.
  A total number of two exhaust sets will be installed.  The amount of supplied
  air is designed to ensure 10 air exchanges per hour.

}

\twoColls{
  Návrhové parametry ventilátoru, jsou uvedeny v příloze: TABLE EQUIPMENT AND
  PRINCIPLE DIAGRAMS.

}{
  Fan design parameters are listed in annex: TABLE EQUIPMENT AND PRINCIPLE
  DIAGRAMS.

}
