\section{Equipment No.09.06-1	- Production area}

\twoColls{
  Popis tohoto zařízení je shodný jako Equipment No.09.01-1- Production area.

}{
  Description of this equipment coincides with that of Equipment No.09.01-1 -
  Production area.

}

\twoColls{
  Návrhové parametry jednotky, jsou uvedeny v příloze: TABLE EQUIPMENT AND
  PRINCIPLE DIAGRAMS.

}{
  Unit design  parameters are listed in annex: TABLE EQUIPMENT AND PRINCIPLE
  DIAGRAMS.

}

\clearpage
\section{Equipment No.09.06-2	- Raw material room}

\twoColls{
  Popis tohoto zařízení je shodný jako Equipment No.09.02-2- Raw material room.
  Změna je pouze v tom, že jednotka je umístěna ve 3.NP.

}{
  Description of this equipment coincides with that of Equipment No.09.02-2 -
  Raw material room except the fact, that AHU for this room will be placed on
  the 3rd floor.

}

\twoColls{
  Návrhové parametry jednotky, jsou uvedeny v příloze: TABLE EQUIPMENT AND
  PRINCIPLE DIAGRAMS.

}{
  Unit design  parameters are listed in annex: TABLE EQUIPMENT AND PRINCIPLE
  DIAGRAMS.

}

\clearpage
\section{Equipment No.09.06-3	- Test and training room}

\twoColls{
  Větrání, chlazení a vytápění Test and training room a MLD material room
  zajišťuje jedna vzduchotechnická jednotka osazená ve strojovně vzduchotechniky v
  2NP.  Teplotně upravovaný venkovní vzduch je přiváděn potrubím do prostoru, kam
  je vyfukován pomocí vyústek. Přívodní vyústky jsou ovládány servomotory na 24V,
  dle prostorvé teploty z každé místnosti.  Odvod vzduchu je pomocí odsávacího
  potrubí, vedeného pod stropem větraného prostoru. 

}{
  Ventilation, cooling and heating of Test and training and MLD material rooms
  will be ensured by single AHU placed in AC machinery room on the 2nd floor.
  Thermally conditioned outdoor air will be supplied through ventilation duct to
  indoor areas by diffusers. Inlet diffusers will be controlled by 24~V
  servomotors according to room temperature. Waste air will be exhausted by
  ducts leading below the ceiling of ventilated area. 

}

\twoColls{
  Jelikož je jedním zařízení větráno více prostor a dále jsou zde umístěny
  technologické odtahy budou na přívodním a odvodním potrubí osazeny regulátory
  proměnného průtoku CAV a VAV. 

}{
  Since many areas will be ventilated by one central equipment, including
  exhaust of heat load from technology, supply and exhaust ducts will be
  equipped with controllers of variable air-flow rate CAV and VAV.

}

\twoColls{
  Regulátor průtoku VAV na přívodu v MLD material room bude regulován dle
  prostorového teplotního čidla. Regulátor  VAV na odvodu bude regulován souběžně
  s regulátorem  VAV na přívodu. 

}{
  VAV controller in supply duct terminating in MLD material room will be set
  according to room temperature sensor. As soon as this controller is set,
  controller settings on outlets will be determined.

}

\twoColls{
  Jelikož jsou v místnosti lokální odtahy od technologie, bude VAV regulátor na
  odtahu ovládán ještě na základě konstantního podtlaku max. 50Pa v místnosti,
  popřípadě dle chodu odsávacích ventilátorů technologie. Otáčky ventilátorů
  v centrální jednotce budou řízeny na konstantní tlak v potrubí. 

}{
  Since there will be a heat load from technology, VAV controller in outlet duct
  will be set in order to ensure constant underpressure in room, max.  50~Pa,
  optionally based on operation of exhaust fans of technology. Fan speeds in
  central unit will be controlled to ensure constant pressure in duct. 

}

\twoColls{
  Množství přiváděného vzduchu je navrženo dle potřeby odvodu tepelné zátěže od
  technologie.

}{
  The amount of supplied air is designed in order to exhaust a heat load from
  technology.

}

\twoColls{
  Vzduchotechnická jednotka je ve vnitřním provedení, opatřena filtrem vzduchu
  (přívod F5+F7 /odtah F5), rotačním výměníkem, směšovací komorou, vodním
  chladičem (15/20\gc), vodním chladičem (9/15\gc), vodním ohřívačem (45/30\gc) a
  dvěma ventilátory s frekvenčními měniči. 

  Zařízení pracuje s nuceným přívodem a odvodem vzduchu. Čerstvý vzduch je nasáván
  VZT jednotkou potrubím z fasády objektu, kde je umístěna protidešťová žaluzie.
  Znehodnocený vzduch je VZT jednotkou vyfukován nad střechu objektu.       

}{
  Innerly designed AHU will be equipped with air filters (F5+F7 supply / F5
  exhaust), rotatory heat exchanger, mixing chamber, water-based air cooler (15
  / 20\gc), water-based air cooler (9 / 15\gc) water-based air heater (45 /
  30\gc) and two fans with frequency changers. 

  Equipment will work with forced
  air inlet and outlet. 
  Fresh air will be suctioned in by AHU through ducts from building facade,
  where a weather resistant louver will be placed.  Waste air will be blown by
  AHU over the roof of the building.

}

\twoColls{
  Návrhové parametry jednotky, jsou uvedeny v příloze: TABLE EQUIPMENT AND
  PRINCIPLE DIAGRAMS.

}{
  Unit design parameters are listed in annex: TABLE EQUIPMENT AND PRINCIPLE
  DIAGRAMS.

}

\clearpage
\section{Equipment No.09.06-4	- Training office}

\twoColls{
  Popis tohoto zařízení je shodný jako Equipment No.09.04-9- MAF + MQ Office.

}{
  Description of this equipment coincides with that of Equipment No.09.04-9 -
  MAF + MQ Office.

}

\twoColls{
  Návrhové parametry jednotky, jsou uvedeny v příloze: TABLE EQUIPMENT AND
  PRINCIPLE DIAGRAMS.

}{
  Unit design parameters are listed in annex: TABLE EQUIPMENT AND PRINCIPLE
  DIAGRAMS.

}

\clearpage
\section{Equipment No.09.06-5	- Corridors}

\twoColls{
  Popis tohoto zařízení je shodný jako Equipment No.09.01-4- Corridors. Změna je
  pouze v tom, že jednotka je umístěna ve 3.NP.


}{
  Description of this equipment coincides with that of Equipment No.09.01-4 -
  Corridors except the fact, that AHU for this room will be placed on
  the 3rd floor.

}

\twoColls{
  Návrhové parametry jednotky, jsou uvedeny v příloze: TABLE EQUIPMENT AND
  PRINCIPLE DIAGRAMS.

}{
  Unit design parameters are listed in annex: TABLE EQUIPMENT AND PRINCIPLE
  DIAGRAMS.

}

\clearpage
\section{Equipment No.09.06-6	- Transformators}

\twoColls{
  Popis tohoto zařízení je shodný jako Equipment No.09.01-7- Transformators.

}{
  Description of this equipment coincides with that of Equipment No.09.01-7 -
  Transformators.

}

\twoColls{
  Návrhové parametry jednotky a ventilátoru, jsou uvedeny v příloze: TABLE
  EQUIPMENT AND PRINCIPLE DIAGRAMS.

}{
  Unit and fan design parameters are listed in annex: TABLE EQUIPMENT AND PRINCIPLE
  DIAGRAMS.

}

\clearpage
\section{Equipment No.09.06-7	- Inverters Switch rooms}

\twoColls{
  Odvod tepelné zátěže z Inverters switch room, bude zajištěn kombinovaným
  systémem nuceného podtlakového odsávání a chlazení vzduchu pomocí cirkulační
  jednotky AHU. V chladném období roku, bude místnost chlazena pouze venkovním
  vzduchem. Venkovní vzduch, bude do místnosti přiváděn z fasády přes filtr G4
  pomocí přívodních ventilátorů s frekvenčním měničem.

}{
  An exhaust of heat load from Invert switch room will be ensured by a
  circulation AHU with combined system of forced vacuum suction and air cooling.
  During a cold year season, the room will be cooled by outdoor air only, which
  will be suctioned in by supply fans with frequency changes from a building
  facade through a duct provided with G4 filter.

}

\twoColls{
  V teplém období roku, budou odsávací střešní ventilátory vypnuty a tepelná zátěž
  z místnosti bude odváděna pomocí cirkulačních jednotek AHU s vodním chladičem.

}{
  During a warm year season,
  an exhaust roof fan will be turned off and a heat load from the room
  will be exhausted by the circulation AHU with a water-based air cooler.

}

\twoColls{
  Vzduchotechnická jednotka je ve vnitřním provedení, opatřena filtrem vzduchu G4,
  vodním chladičem (15/20\gc) a ventilátorem s frekvenčním měničem. 

}{
  Innerly designed AHU will be equipped with G4 air filter, water-based air cooler
  (15 / 20\gc) and fan with frequency changer.

}

\twoColls{
  Návrhové parametry jednotky a ventilátoru, jsou uvedeny v příloze: TABLE
  EQUIPMENT AND PRINCIPLE DIAGRAMS.

}{
  Unit and fan design parameters are listed in annex: TABLE EQUIPMENT AND PRINCIPLE
  DIAGRAMS.

}

\clearpage
\section{Equipment No.09.06-8	- Switch rooms}

\twoColls{
  Odvod tepelné zátěže z switch rooms, bude zajištěn kombinovaným systémem
  nuceného podtlakového odsávání a chlazení vzduchu pomocí cirkulační jednotky
  AHU. V chladném období roku, bude místnost chlazena pouze venkovním vzduchem.
  Venkovní vzduch, bude do místnosti nasáván z fasády přes filtr G4. V teplém
  období roku, bude odsávací střešní ventilator vypnut a tepelná zátěž z místnosti
  bude odváděna pomocí cirkulační jednotky AHU s vodním chladičem.

}{
  An exhaust of heat load from switch rooms will be ensured by a circulation AHU
  with combined system of forced vacuum suction and air cooling. During a cold
  year season, the switch room will be cooled by outdoor air only, which will be
  suctioned in from a building facade through a G4 filter. During a warm year season,
  an exhaust roof fan will be turned off and a heat load from the switch room
  will be exhausted by the circulation AHU with a water-based air cooler.

}

\twoColls{
  Vzduchotechnická jednotka je ve vnitřním provedení, opatřena filtrem vzduchu G4,
  vodním chladičem (15/20\gc) a ventilátorem s frekvenčním měničem. 

}{
  Innerly designed AHU will be equipped with G4 air filter, water-based air cooler
  (15 / 20\gc) and fan with frequency changer.

}

\twoColls{
  Pro každý switch room, je navrženo samostatné odsávací a chladící zařízení. 

}{
  Separate exhaust and cooling systems were devised for each switch room.

}

\twoColls{
  Návrhové parametry jednotky a ventilátoru, jsou uvedeny v příloze: TABLE
  EQUIPMENT AND PRINCIPLE DIAGRAMS.

}{
  Unit and fan design parameters are listed in annex: TABLE EQUIPMENT AND PRINCIPLE
  DIAGRAMS.

}

\clearpage
\section{Equipment No.09.06-9	- AC machinery room}

\twoColls{
  Popis tohoto zařízení je shodný jako Equipment No. 09.01-9- AC machinery room.

}{
  Description of this equipment coincides with that of Equipment No.09.01-9 -
  AC machinery room.

}

\twoColls{
  Návrhové parametry ventilátoru, jsou uvedeny v příloze: TABLE
  EQUIPMENT AND PRINCIPLE DIAGRAMS.

}{
  Fan design parameters are listed in annex: TABLE EQUIPMENT AND PRINCIPLE
  DIAGRAMS.

}

\clearpage
\section{Equipment No.09.06-10 - Engine room of valve station}

\twoColls{
  Větrání technických místností s minimální tepelnou zátěží od technologie, je
  řešeno systémem podtlakového odsávání pomocí potrubních ventilátorů. Ventilátory
  se spouští na základě prostorového teplotního čidla a při vstupu do místnosti.  

}{
  Ventilation of technical rooms having minimal heat load from technology will be
  ensured by a system of vacuum suction by duct fans. Fans will be triggered
  according to room temperature sensor and when a person enters the room.
}

\twoColls{
  Nasávání venkovního vzduchu do technických místností je přes protidešťovou
  žaluzii, klapky a filtry G4.

}{
  Outdoor air are will be supplied to engine rooms through weather resistant louver,
  dampers and G4 filter.

}

\twoColls{
  Množství odváděného vzduchu je navrženo na 1 výměnu vzduchu za hodinu na celou
  výšku místnosti.

}{
  Amount of exhaust air is designed to ensure 1 air exchange per hour throughout
  room height.

}

\twoColls{
  Návrhové parametry ventilátoru, jsou uvedeny v příloze: TABLE
  EQUIPMENT AND PRINCIPLE DIAGRAMS.

}{
  Fan design parameters are listed in annex: TABLE EQUIPMENT AND PRINCIPLE
  DIAGRAMS.

}


\clearpage
\section{Equipment No.09.06-11 - 20kV switch room}

\twoColls{
  Větrání 20kV switch room s minimální tepelnou zátěží od technologie, je řešeno
  systémem podtlakového odsávání pomocí potrubního ventilátoru. Ventilátor se
  spouští na základě prostorového teplotního čidla a při vstupu do místnosti.  

}{
  Ventilation of 20kV switch room having minimal heat load from technology will be
  ensured by a system of vacuum suction by duct fan. Fan will be triggered
  based on room temperature sensor and when a person enters the room.

}

\twoColls{
  Nasávání venkovního vzduchu do technické místnosti je přes protidešťovou
  žaluzii, klapku a filtr G4.

}{
  Outdoor air are will be supplied to switch room through weather resistant louver,
  damper and G4 filter. 

}

\twoColls{
  Chlazení místnosti switch room na předepsanou teplotu zajišťuje profese CHLAZENÍ.

}{
  Cooling of switch room at a prescribed temperature is ensured by profession of
  COOLING.

}

\twoColls{
  Množství odváděného vzduchu je navrženo na 1 výměnu vzduchu za hodinu na celou výšku místnosti.

}{
  Amount of exhaust air is designed to ensure 1 air exchange per hour throughout
  room height.

}

\twoColls{
  Návrhové parametry ventilátoru, jsou uvedeny v příloze: TABLE
  EQUIPMENT AND PRINCIPLE DIAGRAMS.

}{
  Fan design parameters are listed in annex: TABLE EQUIPMENT AND PRINCIPLE
  DIAGRAMS.

}

\clearpage
\section{Equipment No.09.06-12 - Sanitary room}

\twoColls{
  Hygienické místnosti, jsou větrány podtlakovým způsobem pomocí samostatného
  odsávacího zařízení složeného z potrubí, odsávacího ventilátoru umístěného ve
  strojovně AC v 2.NP. a přetlakové klapky. 
  
  Znehodnocený vzduch je z jednotlivých
  místností odsáván pomocí vyústek osazených do podhledu a je ventilátorem
  vyfukován na střechu objektu. Náhradní vzduch je přisáván přes stěnové mřížky a
  pod dveřmi z prostoru chodby.

}{
  Sanitary rooms will be ventilated by an underpressure way by an usage of
  separate suction device composed of ducts, exhaust fan placed in AC machinery
  room on the 2nd floor and overpressure valve. 
  
  Waste air will be suctioned out
  from individual rooms with diffusers mounted in the suspended ceiling and
  blown by ventilator over the roof of the building. Replacement air will be
  suctioned in through wall grilles and from corridor through space under the
  door.

}

\twoColls{
  Návrhové parametry ventilátoru, jsou uvedeny v příloze: TABLE
  EQUIPMENT AND PRINCIPLE DIAGRAMS.

}{
  Fan design parameters are listed in annex: TABLE EQUIPMENT AND PRINCIPLE
  DIAGRAMS.

}

\clearpage
\section{Equipment No.09.06-13 - Protected escape route}

\twoColls{
  Větrání chráněné únikové cesty (CHÚC) je řešeno nuceným přetlakovým způsobem.
  Venkovní vzduch je nasáván z fasády objektu a pomocí  potrubního ventilátoru, je
  vyfukován do prostoru CHÚC. 

}{
  Protected escape routes (PER) will be ventilated by a forced overpressure way.
  Outdoor air will be sucked in from the building facade and by duct fan led to
  the areas of PER. 

}

\twoColls{
  Odvod vzduchu z větraného prostoru je zajištěn vlivem přetlaku v nejvyšším místě
  CHÚC. Odvodní otvor bude opatřen uzavírací a přetlakovou klapkou, na které bude
  nastaven požadovaný přetlak cca 25Pa. Celkem budou použity 3ks výfukových
  kompletů. Přiváděné množství vzduchu do CHÚC zajistí 10 výměn vzduchu za hodinu.

}{
  Air exhaust from ventilated area will be ensured by overpressure in the
  highest point of PER. The outlet will be provided with an overpressure
  shut-off damper, which will be set to ensure the overpressure of ca. 25~Pa.
  Totally 3 pcs of exhaust equipment will be used.  The amount of supply air
  delivered to PER will guarantee 10 air exchanges per hour.

}

\twoColls{
  Návrhové parametry ventilátoru, jsou uvedeny v příloze: TABLE
  EQUIPMENT AND PRINCIPLE DIAGRAMS.

}{
  Fan design parameters are listed in annex: TABLE EQUIPMENT AND PRINCIPLE
  DIAGRAMS.

}
