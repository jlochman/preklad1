\clearpage
\section{Equipment No.09.01-1	- Production area}

\twoColls{
  Větrání a chlazeni prostoru výrobní haly, bude zajištěno soustavou 10 ks
  centrálních jednotek složených z 8 ks AHU jednotek a 2 ks MAU jednotek, které
  budou umístěny v centrální strojovně vzduchotechniky.

}{
  Ventilation and cooling of the production area will be provided by a system of
  10~pcs of central AHU composed of 8~pcs of AHU and 2~pcs of MAU, which will
  be placed in the central AC machinery room.
}

\twoColls{
  Teplotně upravovaný venkovní vzduch je přiváděn potrubím do prostoru, kam je
  vyfukován pomocí  válcových perforovaných textilních vyústek, jež jsou umístěny
  u sloupu, 0,5~m nad podlahou. Odvod vzduchu je pomocí odsávacího potrubí,
  vyústěného pod stropem haly.

}{
  Thermally conditioned outdoor air will be forced through ventilation duct,
  where it will be blown out using cylindrical perforated textile diffusers placed
  next to pillar and in a distance of 0.5~m above the floor. Air exhaust will be
  handled by an exhaust duct below the ceiling.
}

\twoColls{
  Množství přiváděného a odváděného vzduchu je navrženo na odvod celkové tepelné
  zátěže z lisovny 1450 kW. Tomu odpovídá přívodní množství vzduchu 336.000m3/h a
  výměna vzduchu 18 za hodinu do výšky místnosti 3m. 

}{
  The amount of supply and exhaust air is designed to ensure the heat sink of
  1450~kW, which is the heat load from moulding. It corresponds to inlet air flow
  rate of 336,000~m3/h or to 18 air exchanges per hour up to 3~m of room height.  
}

\twoColls{
  AHU jednotky, jsou složeny z filtru F5+F7, přívodního a odsávacího ventilátoru s
  s frekvenčními měniči, směšovací komory a vodního chladiče vzduchu (15 / 20\gc).

}{
  AHUs will be composed of F5+F7 filter, supply and exhaust fans equipped with
  frequency changers, mixing chamber and water-based air cooler (15 / 20\gc).
}

\twoColls{
  V jedné AHU jednotce, je navíc osazen vodní ohřívač (45 / 30\gc) pro teplovzdušné
  vytápění při odstávce výroby. Aby došlo k promíchání teplého vzduchu po celém
  prostoru výrobní haly, budou ostatní AHU jednotky v provozu pouze v cirkulačním
  režimu. 

}{
  One AHU will be equipped with a water-based air heater (45 / 30\gc) to
  provide hot-air during a production shutdown. To distribute hot
  air throughout the production hall area, remaining AHUs will be running in
  circulation mode.  
}

\twoColls{
  Při standardní výrobě, budou AHU jednotky v chladném období roku v provozním
  režimu free cooling, pokud entalpie venkovního vzduchu bude menší, než entalpie
  vzduchu odsávaného.

}{
  During production in cold year season, AHUs will be operating in a free
  cooling mode, if enthalpy of outdoor air is smaller than enthalpy of exhausted
  air.
}

\twoColls{
  V teplém období roku, kdy entalpie venkovního vzduchu bude větší, než entalpie
  vzduchu odsávaného, bude v jednotce AHU vypnut odsávací ventilátor a v provozu,
  bude pouze přívodní část jednotky AHU. Do přívodní části AHU, bude přisáváno
  20\% venkovního vzduchu, který bude upravován v centrální jednotce MAU. Jednotka
  MAU bude zajišťovat úpravu venkovního vzduchu ve filtrech F5+F7, rotačním
  rekuperátoru a ve dvou vodních chladičích vzduchu (15 / 20\gc) + (9 / 15\gc). vč.
  kondenzačního odvlhčování. Motory ventilátorů jsou vybaveny frekvenčními měniči. 

}{
  During production in warm year season, enthalpy of outdoor air will be
  greater than enthalpy of exhausted air.  The AHU outlet will be closed,
  while the inlet remains open.  20~\% of outdoor air will be suctioned into AHU
  inlet. This air will be conditioned in central MAU by F5+F7 filters, rotatory
  heat exchanger and two water-based air coolers (15 / 20\gc) + (9 / 15\gc)
  incl. condensing dehumidification.  Motors of fans will be equipped with
  frequency changers. 
}

\twoRows{
  \begin{center}
    \begin{tabular}{ l l }
      Celkové množství vzduchu přívod/odvod	&		336.000m3/h /320.000m3/h \\
      Počet jednotek AHU			&			8 \\
      Počet jednotek MAU			&			2	 \\
      počet výměn do výšky 3m	&				18/h \\
      podíl venkovního vzduchu v letním období	&		20\% \\
      počet osob na směnu			&			29\\
      počet lisů				&			192\\
      tepelná zátěž od lisu				&		6 kW\\
      výpočtová teplota přiváděného vzduchu	&		+17\gc\\
      výpočtová teplota  vzduchu ve 3m			&	+28\gc\\
      výpočtová teplota odsávaného vzduchu	&			+30\gc\\
    \end{tabular}
  \end{center}
}{
  \begin{center}
    \begin{tabular}{ l l }
      Total amount of air inlet/exhaust	&		336,000~m3/h / 320,000~m3/h \\
      Number of AHUs			&			8 \\
      Number of MAUs			&			2	 \\
      Number of air exchanges &				18/h (up to 3~m of room height) \\
      Outside/total air ratio in summer season	&		20~\% \\
      Number of persons per one shift &			29\\
      Number of moulds				&			192\\
      Heat load from one mould &		6~kW\\
      Calculated temperature of inlet air	&		+17\gc\\
      Calculated temperature of air in working area 		&	+28\gc~(in height of 3~m) \\
      Calculated temperature of exhaust air	&			+30\gc\\
    \end{tabular}
  \end{center}
}


\twoColls{
  Návrhové parametry jednotky, jsou uvedeny v příloze: TABLE EQUIPMENT AND
  PRINCIPLE DIAGRAMS.

}{
  Unit design parameters are listed in annex: TABLE EQUIPMENT AND PRINCIPLE
  DIAGRAMS.
}


\clearpage
\section{Equipment No.09.01-2	- Breaking room}


\twoColls{
  Větrání breaiking room a corridor, zajišťuje jedna vzduchotechnická jednotka
  osazená ve strojovně vzduchotechniky v 2NP. Teplotně upravovaný venkovní vzduch
  je přiváděn potrubím do vnitřních prostor pomocí vířivých anemostatů, jež jsou
  osazeny do podhledu, a znehodnocený vzduch je odsáván potrubím vedeným nad
  podhledem. Propojení mezi větranou místností a odsávaným podhledem, bude
  zajištěno mřížovanou podhledovou kazetou. 

}{
  Breaking room and corridor areas will be ventilated by single AHU placed in AC
  machinery room on the 2nd floor. Thermally conditioned outdoor air will be supplied
  through ventilation duct to indoor areas by swirl diffusers mounted in
  suspended ceiling. Waste air will be exhausted by ducts leading over the suspended
  ceiling. Connection between ventilated room and suctioned suspended ceiling
  will be ensured by perforated ceiling grids

}

\twoColls{
  Tepelný výkon jednotky bude řízen dle prostorového čidla teploty v breaking
  room. Otáčky ventilátoru v centrální jednotce budou řízeny dle konstantního
  tlaku v přívodním a odsávacím potrubí. Poměr směšování vzduchu v centrální VZT
  jednotce, bude řízen dle čidla kvality vzduchu umístěného v breaking room.
  Přiváděná dávka venkovního vzduchu do breaking room je 50m3/h na osobu, nebo
  min. 4x/h.

}{
  Thermal power of AHU will be controlled by a room temperature sensor placed in
  breaking room. Fan speed in central AHU will be set to ensure constant pressure
  in supply and exhaust ducts. Air mixing ration in central AHU will be
  controlled by an air quality sensor placed in breaking room. The amount of
  outdoor air supplied to breaking room will be 50~m3/h per person or at least 4
  air exchanges per hour. 
}

\twoColls{
  Vzduchotechnická jednotka je ve vnitřním provedení, opatřena filtrem vzduchu
  (přívod F5+F7 /odtah F5), rotačním výměníkem, směšovací komorou, vodním
  chladičem (15 / 20\gc), vodním ohřívačem (45 / 30\gc) a dvěma ventilátory
  s frekvenčními měniči. Zařízení pracuje s nuceným přívodem a odvodem vzduchu.
  Čerstvý vzduch je nasáván VZT jednotkou potrubím z fasády objektu, kde je
  umístěna protidešťová žaluzie. Od jednotky je vzduch veden VZT potrubím pod
  stropem jednotlivých větraných prostor. Znehodnocený vzduch je VZT jednotkou
  vyfukován nad střechu objektu.       

}{
  Innerly designed AHU will be equipped with air filters (F5+F7 supply / F5 exhaust),
  rotatory heat exchanger, mixing chamber, water-based air cooler (15 / 20\gc),
  water-based air heater
  (45 / 30\gc) and two fans with frequency changers. Equipment will work with forced air
  inlet and outlet. Fresh air will be suctioned in by two AHUs through ducts from building
  facade, where a weather resistant louver will be placed. Air from AHUs will be led by
  ventilation ducts below the ceiling of ventilated areas. Waste air will be blown
  by AHU over the roof of the building.

}

\twoColls{
  Návrhové parametry jednotky, jsou uvedeny v příloze: TABLE EQUIPMENT AND PRINCIPLE DIAGRAMS.
}{
  Unit design parameters are listed in annex: TABLE EQUIPMENT AND PRINCIPLE DIAGRAMS.
}


\clearpage 
\section{Equipment No.09.01-3	- Equipment room}

\twoColls{
  Větrání, chlazení a vytápění Equipment room zajišťuje jedna vzduchotechnická
  jednotka osazená ve strojovně vzduchotechniky ve 1NP. Teplotně upravovaný
  venkovní vzduch je přiváděn potrubím do prostoru, kam je vyfukován pomocí
  vyústek. Odvod vzduchu je pomocí odsávacího potrubí, vedeného pod stropem
  větraného prostoru.

}{
  Ventilation, cooling and heating of Equipment room will be ensured by single AHU
  placed in AC machinery room on the 1st floor. Thermally conditioned outdoor air
  will be forced through ventilation duct to equipment room, where it will be
  blown out through grills. Waste air will be suctioned out by ducts leading below the
  ceiling.

}

\twoColls{
  Množství přiváděného vzduchu je navrženo na 4 výměny vzduchu za hodinu do výšky
  místnosti 3m. 

}{
  The amount of supplied air is designed to ensure 4 air exchanges per hour up
  to room height of 3~m.

}

\twoColls{
  Vzduchotechnická jednotka je ve vnitřním provedení, opatřena filtrem vzduchu
  (přívod F5+F7 / odtah F5), rotačním výměníkem, směšovací komorou, vodním
  chladičem (15 / 20\gc), vodním ohřívačem (45 / 30\gc) a dvěma ventilátory
  s frekvenčními měniči. Zařízení pracuje s nuceným přívodem a odvodem vzduchu.
  Čerstvý vzduch je nasáván VZT jednotkou potrubím z fasády objektu, kde je
  umístěna protidešťová žaluzie. Od jednotky je vzduch veden VZT potrubím pod
  stropem jednotlivých větraných prostor. Znehodnocený vzduch je VZT jednotkou
  vyfukován nad střechu objektu.       

}{
  Innerly designed AHU will be equipped with air filters (F5+F7 supply / F5
  exhaust), rotatory heat exchanger, mixing chamber, water-based air cooler (15 / 20\gc),
  water-based air heater (45 / 30\gc) and two fans with frequency changers. Equipment will
  work with forced air inlet and outlet. Fresh air will be suctioned in by 
  AHU through ducts from building facade, where a weather resistant louver will
  be placed. Air from AHU will be led by ventilation ducts below the ceiling of
  ventilated areas. Waste air will be blown out by AHU over the roof of the
  building. 

}

\twoColls{
  Návrhové parametry jednotky, jsou uvedeny v příloze: TABLE EQUIPMENT AND
  PRINCIPLE DIAGRAMS.

}{
  Unit design parameters are listed in annex: TABLE EQUIPMENT AND PRINCIPLE
  DIAGRAMS.

}

\clearpage 
\section{Equipment No.09.01-4	- Corridors}

\twoColls{
  Větrání, chlazení a vytápění prostor chodeb zajišťuje jedna vzduchotechnická
  jednotka osazená ve strojovně vzduchotechniky v 2NP. Teplotně upravovaný
  venkovní vzduch je přiváděn potrubím do prostoru, kam je vyfukován pomocí
  vyústek. Odvod vzduchu je pomocí odsávacího potrubí, vedeného pod stropem
  větraného prostoru.

}{
  Ventilation, cooling and heating of corridors will be ensured by single AHU
  placed in AC machinery room on the 2nd floor. Thermally conditioned outdoor air
  will be forced through ventilation duct to corridors, where it will be blown
  through grills. Waste air will be suctioned out by ducts leading below the
  ceiling of ventilated areas.

}

\twoColls{
  Množství přiváděného vzduchu je navrženo na 1 výměnu vzduchu za hodinu na celou
  výšku místnosti.

}{
  The amount of supplied air is designed to ensure 1 air exchanges per hour
  throughout whole room height.

}

\twoColls{
  Vzduchotechnická jednotka je ve vnitřním provedení, opatřena filtrem vzduchu
  (přívod F5+F7 / odtah F5), rotačním výměníkem, směšovací komorou, vodním
  chladičem (15 / 20\gc), vodním ohřívačem (45 / 30\gc) a dvěma ventilátory
  s frekvenčními měniči. Zařízení pracuje s nuceným přívodem a odvodem vzduchu.
  Čerstvý vzduch je nasáván VZT jednotkou potrubím z fasády objektu, kde je
  umístěna protidešťová žaluzie. Od jednotky je vzduch veden VZT potrubím pod
  stropem jednotlivých větraných prostor. Znehodnocený vzduch je VZT jednotkou
  vyfukován nad střechu objektu.       

}{
  Innerly designed AHU will be equipped with air filters (F5+F7 supply / F5
  exhaust), rotatory heat exchanger, mixing chamber, water-based air cooler (15 / 20\gc),
  water-base air heater (45 / 30\gc) and two fans with frequency changers. Equipment will
  work with forced air inlet and outlet. Fresh air will be suctioned in by AHU
  through ducts from building facade, where a weather resistant louver will be
  placed. Air from AHU will be led by ventilation ducts below the ceiling of
  ventilated areas. Waste air will be blown out by AHU over the roof of the
  building. 

}

\twoColls{
  Návrhové parametry jednotky, jsou uvedeny v příloze: TABLE EQUIPMENT AND
  PRINCIPLE DIAGRAMS.

}{
  Unit design parameters are listed in annex: TABLE EQUIPMENT AND PRINCIPLE
  DIAGRAMS.

}

\clearpage 
\section{Equipment No.09.01-5	- Facility room}


\twoColls{
  Odvod tepelné zátěže z facility room, bude zajištěn kombinovaným systémem
  nuceného podtlakového odsávání a chlazení vzduchu pomocí cirkulační jednotky
  AHU. V chladném období roku, bude místnost chlazena pouze venkovním vzduchem.
  Venkovní vzduch, bude do místnosti nasáván z fasády přes filtr G4. V teplém
  období roku, bude odsávací střešní ventilator vypnut a tepelná zátěž z místnosti
  bude odváděna pomocí cirkulační jednotky AHU s vodním chladičem.

}{
  An exhaust of heat load from facility room will be ensured by circulation AHU
  with combined system of forced vacuum suction and air cooling. During a cold
  year season, the facility room will be cooled by outside air only, which will be
  suctioned into facility room through a duct equipped with G4 filter. During a
  warm year season, an exhaust roof ventilator will be turned off and heat load
  from facility room will be exhausted by the circulation AHU with a water-based
  air cooler.

}

\twoColls{
  Vzduchotechnická jednotka je ve vnitřním provedení, opatřena filtrem vzduchu G4,
  vodním chladičem (15/20\gc) a ventilátorem s frekvenčním měničem. 

}{
  Innerly designed AHU is equipped with G4 air filter, water-based air cooler (15 / 20\gc) and
  fan with frequency changer.

}

\twoColls{
  Pro každou místnost facility room, je navrženo samostatné odsávací a chladící
  zařízení. 

}{
  Separate exhaust and cooling systems were devised for each room in facility
  room area.   

}

\twoColls{
  Návrhové parametry jednotky, jsou uvedeny v příloze: TABLE EQUIPMENT AND
  PRINCIPLE DIAGRAMS.

}{
  Unit design parameters are listed in annex: TABLE EQUIPMENT AND PRINCIPLE
  DIAGRAMS.

}

\clearpage 
\section{Equipment No.09.01-6	- Drying material room}

\twoColls{
  Odvod tepelné zátěže z Drying material room, bude zajištěn kombinovaným systémem
  nuceného přívodu a odvodu vzduchu a systémem chlazení vzduchu pomocí cirkulační
  jednotky AHU. V chladném období roku, bude místnost chlazena pouze venkovním
  vzduchem. Venkovní vzduch, bude do místnosti přiváděn 5ks AHU jednotek
  opatřených filtrem G4+F7 a přívodním ventilátorem s frekvenčním měničem.  Teplý
  vzduch, bude odsáván pod stropem pomocí dvou střešních ventilátorů.  V teplém
  období roku, bude systém free cooling  vypnut a tepelná zátěž z místnosti bude
  odváděna pomocí dvou cirkulačních jednotek AHU s vodním chladičem.

}{
  An exhaust of heat load from drying material room will be ensured by
  circulation AHUs with combined system of forced air supply and exhaust and air
  cooling. During a cold year season, the drying material room will be cooled by
  outdoor air only, which will be supplied by 5 pcs of AHUs equipped with
  G4+F7 filters and supply fan with frequency changer. Warm air will be
  suctioned out from space below the ceiling by two roof fans. During a warm year season,
  a free cooling system will be turned off and the heat load will be exhausted by
  two circulation AHUs with water-based air cooler.
}

\twoColls{
  Vzduchotechnické chladící jednotky jsou ve vnitřním provedení, opatřeny filtrem
  vzduchu G4, vodním chladičem (15 / 20\gc) a ventilátorem s frekvenčním měničem.
  Chladný vzduch je do místnosti přiváděn pomocí válcových perforovaných
  textilních vyústek, jež jsou umístěny po obvodě místnosti 0,5m nad podlahou.

}{
  Innerly designed cooling units will be equipped with G4 air filter,
  water-based air
  cooler (15 / 20\gc) and fan with frequency changer. Cold air will be supplied
  through cylindrical perforated textile diffusers placed around the
  circumference of the romm 0.5~m above the floor.

}

\twoColls{
  Pro technologické lokální odsávání je navrřen centrální střešní ventilátor o
  celkovém vzduchovém výkonu 15.000m3/h.  Na centrálním odsávacím potrubí je 10ks
  odboček na kterých jsou osazeny uzavírací klapky ovládané servomotory. Podle
  počtu otevřených klapek bude regulován výkon centrálního odsávacího ventilátoru.

}{
  For local technological exhaust, a central roof fan with a total air flow rate
  of 15,000 m3/h was devised. There is 10 pcs of digressions on central exhaust
  duct, each equipped with closing damper controlled by servomotor. Power of the
  roof fan will be set according to number of opened dampers.

}


\twoColls{
  Pod stropem haly je zavěšena jedna teplovzdušná cirkulační jednotka napojená na
  centrální rozvod topné vody, jež zajišťuje krytí tepelných ztrát místnosti.
  V případě požadavku na vytápění, je teplý vzduch přiváděn od stropu směrem dolů
  k podlaze. 

}{
  Below the ceiling, there will be suspended one hot-air circulation unit
  connected to central distribution of heat water. This circulation unit will
  ensure coverage of heat losses in the room. In case of heating, hot air will be
  supplied from ceiling down to the floor.

}


\twoColls{
  Návrhové parametry jednotky, jsou uvedeny v příloze: TABLE EQUIPMENT AND
  PRINCIPLE DIAGRAMS.

}{
  Unit design parameters are listed in annex: TABLE EQUIPMENT AND PRINCIPLE
  DIAGRAMS.

}


\clearpage 
\section{Equipment No.09.01-7 - Transformators}

\twoColls{
  Odvod tepelné zátěže z prostoru transformátoru, bude zajištěn kombinovaným
  systémem nuceného podtlakového odsávání a chlazení vzduchu pomocí cirkulační
  jednotky AHU. V chladném období roku, bude transformátor chlazen pouze venkovním
  vzduchem.  Venkovní vzduch bude do místnosti transformátoru nasáván přes
  žaluziové dveře, opatřené na vnitřní straně filtrační tkaninou G3. V teplém
  období roku, bude odsávací střešní ventilator vypnut a tepelná zátěž od
  transformátoru bude odváděna pomocí cirkulační jednotky AHU s vodním chladičem.

}{
  An exhaust of heat load from transformer space will be ensured by a circulation AHU
  with combined system of forced vacuum suction and air cooling. During a cold
  year season, the transformer space will be cooled by outdoor air only, which
  will be suctioned into transformer space through louvered doors provided with
  the cloth filter G3 on its inner side. During a warm year season, an exhausted
  roof fan will be turned off and a heat load from transformers will be
  exhausted by the circulation AHU with water-based air cooler.

}

\twoColls{
  Vzduchotechnická jednotka je ve vnitřním provedení, opatřena filtrem vzduchu G4,
  vodním chladičem (15/20\gc) a ventilátorem s frekvenčním měničem. 

}{
  Innerly designed AHU will be equipped with G4 air filter, water-based air cooler (15
  / 20\gc) and fan with frequency changer.


}

\twoColls{
  Pro každý transformátor je vždy navrženo sa\-mo\-stat\-né odsávací  a chladící
  zařízení.

}{
  Separate exhaust and cooling systems were devised for each transformer.

}

\twoColls{
  Návrhové parametry jednotky, jsou uvedeny v příloze: TABLE EQUIPMENT AND
  PRINCIPLE DIAGRAMS.

}{
  Unit design parameters are listed in annex: TABLE EQUIPMENT AND PRINCIPLE
  DIAGRAMS.

}

\clearpage 
\section{Equipment No.09.01-8	- Switch rooms}

\twoColls{
  Odvod tepelné zátěže z switch rooms, bude zajištěn kombinovaným systémem
  nuceného podtlakového odsávání a chlazení vzduchu pomocí cirkulační jednotky
  AHU. V chladném období roku, bude místnost chlazena pouze venkovním vzduchem.
  Venkovní vzduch, bude do místnosti nasáván z fasády přes filtr G4. V teplém
  období roku, bude odsávací střešní ventilator vypnut a tepelná zátěž z místnosti
  bude odváděna pomocí cirkulační jednotky AHU s vodním chladičem.

}{
  An exhaust of heat load from switch rooms will be ensured by a circulation AHU
  with combined system of forced vacuum suction and air cooling. During a cold
  year season, the switch room will be cooled by outdoor air only, which will be
  suctioned in through a duct provided with G4 filter. During a warm year season,
  an exhausted roof fan will be turned off and a heat load from the switch room
  will be exhausted by the circulation AHU with a water-based air cooler.

}

\twoColls{
  Vzduchotechnická jednotka je ve vnitřním provedení, opatřena filtrem vzduchu G4,
  vodním chladičem (15 / 20\gc) a ventilátorem s frekvenčním měničem. 

}{
  Innerly designed AHU will be equipped with G4 air filter, water-based air cooler
  (15 / 20\gc) and fan with frequency changer.

}

\twoColls{
  Pro každý switch room, je navrženo samostatné odsávací a chladící zařízení. 

}{
  Separate exhaust and cooling systems were devised for each switch room.

}

\twoColls{
  V případě SP INVERTERS SWITCH ROOM, která je umístěna vzdáleně od fasády, jsou
  pro posílení přívodu vzduchu u systém free cooling  použity přívodní ventilátory
  s frekvenčním měničem.

}{
  In case of SP INVERTERS SWITCH ROOM, which is located remotely from the facade,
  air supply from free cooling system is fortified by fans with frequency
  changers. 
}

\twoColls{
  Návrhové parametry jednotky, jsou uvedeny v příloze: TABLE EQUIPMENT AND
  PRINCIPLE DIAGRAMS.

}{
  Unit design parameters are listed in annex: TABLE EQUIPMENT AND PRINCIPLE
  DIAGRAMS.

}

\clearpage 
\section{Equipment No.09.01-9	- AC machinery room}

\twoColls{
  Větrání strojoven vzduchotechniky je řešeno v jednom případě pomocí axiálního
  potrubního ventilátoru a v ostatních případech pomocí střešních ventilátorů.
  Vzduchový výkon je navržen na 3x výměnu vzduchu za hodinu. U axiálního
  ventilátoru je do potrubí osazena klapka a tlumič hluku. Střešní ventilátory
  budou osazeny na hluk tlumícím soklu a budou vybaveny zpětnou klapkou. Náhradní
  vzduch bude přisáván z venkovního prostoru přes protidešťovou žaluzii, uzavírací
  klapku a filtr G4.

}{
  Ventilation of AC machinery rooms will be in the most cases ensured by a roof
  fans. Only in one case, the axial duct fan will be used instead. Air flow rate is
  designed for 3 air exchanges per hour. Duct with axial fan will be equipped with
  damper and silencer. Roof fans will be mounted on noise-absorbing base and
  equipped with back-draft damper. Spare are will be supplied through 
  weather resistant louver, shut-off damper and G4 filter.
}

\twoColls{
  Návrhové parametry ventilátoru, jsou uvedeny v příloze: TABLE EQUIPMENT AND
  PRINCIPLE DIAGRAMS.

}{
  Fan design parameters are listed in annex: TABLE EQUIPMENT AND PRINCIPLE
  DIAGRAMS.

}

\clearpage 
\section{Equipment No.09.01-10 - Engine room of valve station}

\twoColls{
  Větrání technické místností s minimální tepelnou zátěží od technologie, je
  řešeno systémem podtlakového odsávání pomocí potrubního ventilátoru. Ventilátor
  se spouští na základě prostorového teplotního čidla a při vstupu do místnosti.  

}{
  Ventilation of engine room having minimal heat load from technology will be
  ensured by a system of vacuum suction by duct fan. Fan will be triggered based
  on room temperature sensor and when a person enters the room.

}

\twoColls{
  Nasávání venkovního vzduchu do technické místnosti, je přes protidešťovou
  žaluzii, klapku a filtry G4.

}{
  Spare outdoor air are will be supplied through weather resistant louver,
  shut-off damper and G4 filter.

}

\twoColls{
  Návrhové parametry ventilátoru, jsou uvedeny v příloze: TABLE EQUIPMENT AND
  PRINCIPLE DIAGRAMS.

}{
  Fan design parameters are listed in annex: TABLE EQUIPMENT AND PRINCIPLE DIAGRAMS.

}


\clearpage
\section{Equipment No.09.01-11 - 20kV switch room}

\twoColls{
Větrání technické místností s minimální tepelnou zátěží od technologie, je
řešeno systémem podtlakového odsávání pomocí potrubního ventilátoru. Ventilátor
se spouští na základě prostorového teplotního čidla a při vstupu do místnosti.  

}{
  Ventilation of switch room having minimal heat load from technology will be
  ensured by a system of vacuum suction by duct fan. Fan will be triggered
  based on room temperature sensor and when a person enters the room.

}

\twoColls{
Nasávání venkovního vzduchu do technické místnosti, je přes protidešťovou
žaluzii, klapku a filtr G4. Vzduchový výkon je navržen na 1x výměnu vzduchu za
hodinu.

}{
    Spare outdoor air are will be supplied through weather resistant louver,
    shut-off damper and G4 filter. The amount of supplied air is designed to
    ensure 1 air exchange per hour

}

\twoColls{
Návrhové parametry ventilátoru, jsou uvedeny v příloze: TABLE EQUIPMENT AND
PRINCIPLE DIAGRAMS.

}{
  Fan design parameters are listed in annex: TABLE EQUIPMENT AND PRINCIPLE
DIAGRAMS.

}

\twoColls{
Chlazení místnosti na předepsanou teplotu zajišťuje profese CHLAZENÍ.

}{
  Cooling of switch room at a prescribed temperature is ensured by profession of
  COOLING.
  

}

\clearpage
\section{Equipment No.09.01-12 - Sanitary room}

\twoColls{
Hygienické místnosti, jsou větrány podtlakovým způsobem pomocí samostatného
odsávacího zařízení složeného z potrubí, odsávacího ventilátoru umístěného pod
stropem větrané místnosti a přetlakové klapky. Znehodnocený vzduch je
z jednotlivých místností odsáván pomocí vyústek osazených do podhledu a je
ventilátorem vyfukován na střechu objektu. Náhradní vzduch je přisáván přes
stěnové mřížky a pod dveřmi z prostoru chodby.

}{
Sanitary rooms will be ventilated be a vacuum way using a
separate exhaust system consisting of ducts, exhaust fan mounted below the
ceiling and overpressure valve. Waste air from individual sanitary rooms will be sucked through ventilation
valves mounted in the ceiling and blown over the roof by fan.
Spare air will be supplied through wall grilles and through the space under the door
from corridor area. 

}

\twoColls{
Návrhové parametry ventilátoru, jsou uvedeny v příloze: TABLE EQUIPMENT AND
PRINCIPLE DIAGRAMS.

}{
Fan design parameters are listed in annex: TABLE EQUIPMENT AND PRINCIPLE
DIAGRAMS.

}

\clearpage
\section{Equipment No.09.01-13 - Spare room}

\twoColls{
Větrání, chlazení a vytápění Spare room zajišťuje jedna vzduchotechnická
jednotka osazená na plošině pod stropem větrané místnosti. Teplotně upravovaný
venkovní vzduch je přiváděn potrubím do prostoru, kam je vyfukován pomocí
vyústek. Přívodní vyústky jsou ovládány servomotory na 24V, dle prostorvé
teploty.  Odvod vzduchu je pomocí odsávacího potrubí, vedeného pod stropem
větraného prostoru.

}{
Ventilation, cooling and heating of Spare room will be ensured by single AHU
mounted on the platform below the ceiling of ventilated room. Thermally
conditioned outdoor air will be forced through ventilation duct to spare room,
where it will be blown out through grills.  Supply air diffusers will be controlled
by 24~V servomotors according to temperature in the room.  Waste air will be
suctioned out by ducts leading below the ceiling.

}

\twoColls{
Množství přiváděného vzduchu je navrženo na 4 výměny vzduchu za hodinu do výšky
místnosti 3m. 

}{
  The amount of supplied air is designed to ensure 4 air exchanges per hour up
  to room height of 3 m.

}

\twoColls{
Vzduchotechnická jednotka je ve vnitřním provedení, opatřena filtrem vzduchu
(přívod F5+F7 /odtah F5), rotačním výměníkem, směšovací komorou, vodním
chladičem (15 / 20\gc), vodním ohřívačem (45 / 30\gc) a dvěma ventilátory
s frekvenčními měniči. Zařízení pracuje s nuceným přívodem a odvodem vzduchu.
Čerstvý vzduch je nasáván VZT jednotkou potrubím z fasády objektu, kde je
umístěna protidešťová žaluzie. Od jednotky je vzduch veden VZT potrubím pod
stropem větraného prostoru. Znehodnocený vzduch je VZT jednotkou vyfukován nad
střechu objektu.       

}{
  
Innerly designed AHU will be equipped with air filters (F5+F7 supply / F5
exhaust), rotatory heat exchanger, mixing chamber, water-based air cooler (15 /
20\gc), water-based air heater (45 / 30\gc) and two fans with frequency
changers. Equipment will work with forced air inlet and outlet. Fresh air will
be suctioned in by AHU through ducts from building facade, where a weather
resistant louver will be placed. Air from AHU will be led by ventilation ducts
below the ceiling of ventilated area. Waste air will be blown out by AHU over
the roof of the building.

}

\twoColls{
Návrhové parametry jednotky, jsou uvedeny v příloze: TABLE EQUIPMENT AND
PRINCIPLE DIAGRAMS.

}{
  Fan design parameters are listed in annex: TABLE EQUIPMENT AND PRINCIPLE
DIAGRAMS.

}

\clearpage
\section{Equipment No.09.01-14 - Protected escape route}

\twoColls{
Větrání chráněné únikové cesty (CHÚC) je řešeno nuceným přetlakovým způsobem.
Venkovní vzduch je nasáván z fasády objektu a pomocí 2 ks potrubních
ventilátorů, je vyfukován do prostoru CHÚC. 

}{
Protected escape routes (PER) will be ventilated by a forced overpressure way.
Outdoor air will be sucked in from the facade and using led by two duct fans to
the areas of PER. 
  

}

\twoColls{
Odvod vzduchu z větraného prostoru je zajištěn vlivem přetlaku v nejvyšším místě
CHÚC. Odvodní otvor bude na vnitřní straně osazen uzavírací a přetlakovou
klapkou, na které bude nastaven požadovaný přetlak cca 25Pa. Celkem budou
použity 4ks výfukových kompletů. Přiváděné množství vzduchu do CHÚC zajistí 10
výměn vzduchu za hodinu.

}{
  Air exhaust from ventilated area will be ensured by overpressure in the
  highest point of PER. The outlet will be provided with an overpressure
  shut-off valve on its inner side, which is set to ensure the overpressure of
  ca. 25 Pa. Totally 4 pcs of exhaust equipment will be used.  The amount of
  supply air delivered to PER will guarantee 10 air exchanges per hour.
  

}

\twoColls{
Návrhové parametry ventilátoru, jsou uvedeny v příloze: TABLE EQUIPMENT AND
PRINCIPLE DIAGRAMS.

}{
Fan design parameters are listed in annex: TABLE EQUIPMENT AND PRINCIPLE
DIAGRAMS.

}

