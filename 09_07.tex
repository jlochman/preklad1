\section{Equipment No.09.07-1 - Production area}

\twoColls{
  Popis tohoto zařízení je shodný jako Equipment No.09.01-1 - Production area.

}{
  Description of this equipment coincides with that of Equipment No.09.01-1 -
  Production area.

}

\twoColls{
  Návrhové parametry jednotky, jsou uvedeny v příloze: TABLE EQUIPMENT AND
  PRINCIPLE DIAGRAMS.

}{
  Unit design parameters are listed in annex: TABLE EQUIPMENT AND PRINCIPLE DIA-
  GRAMS.

}

\clearpage
\section{Equipment No.09.07-2	- Breaking room}

\twoColls{
  Popis tohoto zařízení je shodný jako Equipment No.09.01-2 - Breaking room.

}{
  Description of this equipment coincides with that of Equipment No.09.01-2 -
  Breaking room.

}

\twoColls{
  Návrhové parametry jednotky, jsou uvedeny v příloze: TABLE EQUIPMENT AND
  PRINCIPLE DIAGRAMS.

}{
  Unit design parameters are listed in annex: TABLE EQUIPMENT AND PRINCIPLE DIA-
  GRAMS.

}

\clearpage
\section{Equipment No.09.07-3	- Facility room}

\twoColls{
  Popis tohoto zařízení je shodný jako Equipment No.09.01-5 - Facility room.

}{
  Description of this equipment coincides with that of Equipment No.09.01-5 -
  Facility room.

}

\twoColls{
  Návrhové parametry jednotky, jsou uvedeny v příloze: TABLE EQUIPMENT AND
  PRINCIPLE DIAGRAMS.

}{
  Unit design parameters are listed in annex: TABLE EQUIPMENT AND PRINCIPLE DIA-
  GRAMS.

}

\clearpage
\section{Equipment No.09.07-4	- Drying material room}

\twoColls{
  Popis tohoto zařízení je shodný jako Equipment No.09.01-6 - Drying material room.

}{
  Description of this equipment coincides with that of Equipment No.09.01-6 -
  Drying material room.

}

\twoColls{
  Návrhové parametry jednotky, jsou uvedeny v příloze: TABLE EQUIPMENT AND
  PRINCIPLE DIAGRAMS.

}{
  Unit design parameters are listed in annex: TABLE EQUIPMENT AND PRINCIPLE DIA-
  GRAMS.

}

\clearpage
\section{Equipment No.09.07-5	- AC machinery room}

\twoColls{
  Větrání strojoven vzduchotechniky je řešeno v jednom případě pomocí axiálního
  potrubního ventilátoru a v druhém případě pomocí střešního ventilátoru.
  Vzduchový výkon je navržen na 3x výměnu vzduchu za hodinu. U axiálního
  ventilátoru je do potrubí osazena klapka a tlumič hluku. 

  Střešní ventilátor bude osazen na hluk tlumícím soklu a bude vybaven zpětnou
  klapkou. Náhradní vzduch bude přisáván z venkovního prostoru přes
  protidešťovou žaluzii, uzavírací klapku a filtr G4.

}{
  Ventilation of AC machinery rooms will be in one case ensured by axial duct
  fan, whereas in second case, roof fan will be used instead.  Air flow rate is
  designed for 3 air exchanges per hour. Duct with axial fan will be equipped
  with damper and silencer. 

  Roof fan will be mounted on noise-absorbing base and equipped with back-draft
  damper. Replacement are will be supplied from outside through weather
  resistant louver, shut-off damper and G4 filter.

}

\twoColls{
  Návrhové parametry ventilatoru, jsou uvedeny v příloze: TABLE EQUIPMENT AND
  PRINCIPLE DIAGRAMS.

}{
  Fan design parameters are listed in annex: TABLE EQUIPMENT AND PRINCIPLE DIA-
  GRAMS.

}

\clearpage
\section{Equipment No.09.07-6	- Sanitary room}

\twoColls{
  Popis tohoto zařízení je shodný jako Equipment No.09.01-12 - Sanitary room.

}{
  Description of this equipment coincides with that of Equipment No.09.01-12 -
  Sanitary room.

}

\twoColls{
  Návrhové parametry jednotky, jsou uvedeny v příloze: TABLE EQUIPMENT AND
  PRINCIPLE DIAGRAMS.

}{
  Unit design parameters are listed in annex: TABLE EQUIPMENT AND PRINCIPLE DIA-
  GRAMS.

}

\clearpage
\section{Equipment No.09.07-7	- Switch rooms – reserve}

\twoColls{
  Odvod tepelné zátěže ze switch room, bude zajištěn kombinovaným systémem
  nuceného podtlakového odsávání a chlazení vzduchu pomocí cirkulační jednotky
  AHU. V chladném období roku, bude místnost chlazena pouze venkovním vzduchem.
  Venkovní vzduch, bude do místnosti nasáván z fasády přes filtr G4. V teplém
  období roku, bude odsávací střešní ventilator vypnut a tepelná zátěž z místnosti
  bude odváděna pomocí cirkulační jednotky AHU s vodním chladičem.

}{
  An exhaust of heat load from switch rooms will be ensured by a circulation AHU
  with combined system of forced vacuum suction and air cooling. During a cold
  year season, the switch room will be cooled by outdoor air only, which will be
  suctioned in through a duct provided with G4 filter. During a warm year season,
  an exhaust roof fan will be turned off and a heat load from the switch room
  will be exhausted by the circulation AHU with a water-based air cooler.

}

\twoColls{
  Vzduchotechnická jednotka je ve vnitřním provedení, opatřena filtrem vzduchu G4,
  vodním chladičem (15/20\gc) a ventilátorem s frekvenčním měničem. 

}{
  Innerly designed AHU will be equipped with G4 air filter, water-based air cooler
  (15 / 20\gc) and fan with frequency changer.

}

\twoColls{
  Návrhové parametry jednotky a ventilátoru, jsou uvedeny v příloze: TABLE
  EQUIPMENT AND PRINCIPLE DIAGRAMS.

}{
  Unit and fan design parameters are listed in annex: TABLE EQUIPMENT AND PRINCIPLE DIA-
  GRAMS.

}

\clearpage
\section{Equipment No.09.07-8	- Spare room}

\twoColls{
  Popis tohoto zařízení je shodný jako Equipment No.09.01-13 - Spare room.

}{
  Description of this equipment coincides with that of Equipment No.09.01-13 -
  Spare room.

}

\twoColls{
  Návrhové parametry jednotky, jsou uvedeny v příloze: TABLE EQUIPMENT AND
  PRINCIPLE DIAGRAMS.

}{
  Unit design parameters are listed in annex: TABLE EQUIPMENT AND PRINCIPLE DIA-
  GRAMS.

}
