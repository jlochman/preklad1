\section{Equipment No.09.09-1 - Machinery room}

\twoColls{
Větrání strojovny je zajištěno pomocí čtyř střešních ventilátorů, každý o
vzduchovém výkonu 20.000m3/h, tj. celkem 80.000m3/h. Vzduchový výkon je navržen
na odvod celkových tepelných zisků 126 kW a vytváří v místnosti 4,8x výměnu
vzduchu za hodinu. Ventilátory budou osazeny na hluk tlumícím soklu a budou
vybaveny zpětnou klapkou. Náhradní vzduch bude přisáván z venkovního prostoru
přes protidešťové žaluzie, uzavírací klapky a filtry G4. odsávací ventilátory se
budou pouštět postupně, na základě teplotních čidel rozmístěných ve strojovně. 

}{

}

\twoColls{
Dále jsou do strojovny navrženy další dva odsávací ventilátory pro odsávání
zápachu při nakládání kalových briket do nákladního automobilu. Jeden nástěnný
axiální ventilátor o vzduchovém výkonu 10.000m3/h je osazen do fasády pod
plošinou kalolisu a druhý střešní ventilátor o vzduchovém výkonu 10.000m3/h je
na střeše, půdorysně v místě kalolisu. Oba odsávací ventilátory budou spouštěny
ručně vypínačem umístěným vedle vrat.

}{

}

\twoColls{
V prostoru strojovny jsou umístěny čtyři vany s teplotou vody 25\gc, o celkové
vodní ploše 43,2m2. Při vnitřní teplotě vzduchu +10\gc je předpoklad odparu vody
cca 7 kg/h.  V zimním období stačí pro odvod vlhkostní zátěže vzduchový výkon
7.000m3/h. Aby se strojovna nepodchlazovala, bude vzduchový výkon odsávacího
ventilátoru řízen plynule od prostorových čidel relativní vlhkosti.

}{

}

\twoColls{
Návrhové parametry ventilátorů, jsou uvedeny v příloze: TABLE EQUIPMENT AND
PRINCIPLE DIAGRAMS.

}{

}




\clearpage
\section{Equipment No.09.09-2 - Switch room}

\twoColls{
Větrání a odvod tepelné zátěže z switch room, bude zajištěno kombinovaným
systémem nuceného podtlakového odsávání a chlazení vzduchu pomocí cirkulační
jednotky SPLIT. V chladném období roku, bude místnost chlazena pouze venkovním
vzduchem. Venkovní vzduch, bude do místnosti nasáván z fasády přes protidešťovou
žaluzii, uzavírací klapku a filtr G4.  Vzduch z místnosti bude odsáván pomocí
střešního ventilátoru osazeném na hluk tlumícím soklu se zpětnou klapkou.

}{

}

\twoColls{
Vzduchový výkon ventilátoru 1200m3/h zajistí v místnosti 11x výměnu vzduchu za
hodinu. V teplém období roku, bude odsávací střešní ventilator vypnut a tepelná
zátěž z místnosti bude odváděna pomocí cirkulační chladící jednotky SPLIT
(jednotka SPLIT je předmětem projektu chlazení).

}{

}

\twoColls{
Návrhové parametry ventilátoru, jsou uvedeny v příloze: TABLE EQUIPMENT AND
PRINCIPLE DIAGRAMS.

}{

}





\clearpage
\section{Equipment No.09.09-3 - Storage of spare parts}

\twoColls{
Větrání skladu je řešeno pomocí jednoho střešního ventilátoru. Vzduchový výkon
je navržen na 4x výměny vzduchu za hodinu. Ventilátor bude osazen na hluk
tlumícím soklu a bude vybaven zpětnou klapkou. Náhradní vzduch bude přisáván
z venkovního prostoru přes sestavu složenou  z protidešťové žaluzie, uzavírací
klapky a filtru G4.

}{

}

\twoColls{
Návrhové parametry ventilátoru, jsou uvedeny v příloze: TABLE EQUIPMENT AND
PRINCIPLE DIAGRAMS.

}{

}



\clearpage
\section{Equipment No.09.09-4 - Demi water station and pumping room}

\twoColls{
Větrání místnosti a odvod tepelné zátěže je řešeno pomocí jednoho střešního
ventilátoru. Vzduchový výkon ventilátoru 3000m3/h zajistí v místnosti 5x výměnu
vzduchu za hodinu. Ventilátor bude osazen na hluk tlumícím soklu a bude vybaven
zpětnou klapkou. Náhradní vzduch bude přisáván z venkovního prostoru přes
sestavu složenou  z protidešťové žaluzie, uzavírací klapky a filtru G4.

}{

}

\twoColls{
Návrhové parametry ventilátoru, jsou uvedeny v příloze: TABLE EQUIPMENT AND
PRINCIPLE DIAGRAMS.

}{

}



\clearpage
\section{Equipment No.09.09-5 - Dosing of flacculant}

\twoColls{
Větrání místnosti a odvod tepelné zátěže je řešeno pomocí jednoho střešního
ventilátoru. Vzduchový výkon ventilátoru 1500m3/h zajistí v místnosti 4,4x
výměnu vzduchu za hodinu. Ventilátor bude osazen na hluk tlumícím soklu a bude
vybaven zpětnou klapkou. Náhradní vzduch bude přisáván z venkovního prostoru
přes sestavu složenou  z protidešťové žaluzie, uzavírací klapky a filtru G4.

}{

}

\twoColls{
Návrhové parametry ventilátoru, jsou uvedeny v příloze: TABLE EQUIPMENT AND
PRINCIPLE DIAGRAMS.

}{

}



\clearpage
\section{Equipment No.09.09-6 - Ventilation}

\twoColls{
Větrání a odvod tepelné zátěže z AC machinery, bude zajištěno kombinovaným
systémem nuceného podtlakového odsávání a chlazení vzduchu pomocí cirkulační
jednotky SPLIT. V chladném období roku, bude místnost chlazena pouze venkovním
vzduchem. Venkovní vzduch, bude do místnosti nasáván z fasády přes protidešťovou
žaluzii, uzavírací klapku a filtr G4.  Vzduch z místnosti bude odsáván pomocí
střešního ventilátoru osazeném na hluk tlumícím soklu se zpětnou klapkou.

}{

}

\twoColls{
Vzduchový výkon ventilátoru 1000m3/h zajistí v místnosti 3x výměnu vzduchu za
hodinu. V teplém období roku, bude odsávací střešní ventilator vypnut a tepelná
zátěž z místnosti bude odváděna pomocí cirkulační chladící jednotky SPLIT
(jednotka SPLIT je předmětem projektu chlazení).

}{

}

\twoColls{
Návrhové parametry ventilátoru, jsou uvedeny v příloze: TABLE EQUIPMENT AND
PRINCIPLE DIAGRAMS.

}{

}


\clearpage
\section{Equipment No.09.09-7 - Laboratory}

\twoColls{
Větrání místností control room a laboratory zajišťuje jedna vzduchotechnická
rekuperační jednotka osazená ve strojovně vzduchotechniky. Teplotně upravovaný
venkovní vzduch je přiváděn potrubím do vnitřních prostor pomocí vířivých
anemostatů, jež jsou osazeny do podhledu, a znehodnocený vzduch je odsáván
potrubím s vyústkami zaústěnými do podhledu. V případě potřeby bude přiváděný
vzduch dohříván v elektrickém ohřívači. Přiváděná dávka venkovního vzduchu je
min. 50m3/h na osobu, nebo min. 4x/h.

}{

}

\twoColls{
Vzduchotechnická jednotka je ve vnitřním provedení, opatřena filtrem vzduchu
(přívod F5+F7 /odtah F5), deskovým výměníkem s obtokem, elektrickým ohřívačem a
dvěma ventilátory s frekvenčními měniči, nebo EC motory. Zařízení pracuje
s nuceným přívodem a odvodem vzduchu. Čerstvý vzduch je nasáván VZT jednotkou
potrubím z fasády objektu, kde je umístěna protidešťová žaluzie. Znehodnocený
vzduch je VZT jednotkou vyfukován nad střechu objektu. 

}{

}

\twoColls{
Chlazení místností zajišťuje profese chlazení pomocí dvou kazetových jednotek
SPLIT.

}{

}

\twoColls{
Návrhové parametry jednotky, jsou uvedeny v příloze: TABLE EQUIPMENT AND
PRINCIPLE DIAGRAMS.

}{

}




\clearpage
\section{Equipment No.09.09-8 - Storage of chemicals}

\twoColls{
Větrání skladu je řešeno pomocí jednoho střešního ventilátoru. Vzduchový výkon
ventilátoru 1500m3/h zajistí v místnosti 6x výměnu vzduchu za hodinu. Ventilátor
bude osazen na hluk tlumícím soklu a bude vybaven zpětnou klapkou. Náhradní
vzduch bude přisáván z venkovního prostoru přes sestavu složenou  z protidešťové
žaluzie, uzavírací klapky a filtru G4.

}{

}

\twoColls{
Návrhové parametry ventilátoru, jsou uvedeny v příloze: TABLE EQUIPMENT AND
PRINCIPLE DIAGRAMS.

}{

}



\clearpage
\section{Equipment No.09.09-9 - Machine room of dosing}

\twoColls{
Větrání místnosti, kde dochází k dávkování dioxinu chlóru, je řešeno nuceným
podtlakovým způsobem, pomocí radiálního ventilátoru. Vzduchový výkon havarijního
větrání je navržen na 10x výměn vzduchu za hodinu. Ventilátor z PVC s odolností
pro odsávání plynného chlóru, bude s motorem mimo proud vzdušiny, osazen na
střeše objektu. Odsávací potrubí bude v místnosti staženo až k podlaze a
odsávání znehodnoceného vzduchu bude zajištěno pomocí tří vyústek umístěných u
podlahy, v polovině výšky a pod stropem. 

}{

}

\twoColls{
Náhradní vzduch bude do místnosti nasáván přes stěnové mřížky z machinery room. 

}{

}

\twoColls{
Návrhové parametry ventilátoru, jsou uvedeny v příloze: TABLE EQUIPMENT AND
PRINCIPLE DIAGRAMS.

}{

}
