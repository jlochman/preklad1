\section{Equipment No.09.09-1 - Machinery room}

\twoColls{
  Větrání strojovny je zajištěno pomocí čtyř střešních ventilátorů, každý o
  vzduchovém výkonu 20.000 m3/h, tj. celkem 80.000 m3/h. Vzduchový výkon je
  navržen na odvod celkových tepelných zisků 126 kW a vytváří v místnosti 4,8x
  výměnu vzduchu za hodinu. 

  Ventilátory budou osazeny na hluk tlumícím soklu a
  budou vybaveny zpětnou klapkou. Náhradní vzduch bude přisáván z venkovního
  prostoru přes protidešťové žaluzie, uzavírací klapky a filtry G4. odsávací
  ventilátory se budou pouštět postupně, na základě teplotních čidel rozmístěných
  ve strojovně. 

}{
  Ventilation of Machinery Room will be provided with four roof fans, each having
  air flow rate of 20,000 m3/h (i.e. totally 80,000 m3/h). Air flow rate is
  designed to ensure exhaust of overall 126 kW of heat gain, which corresponds to
  4.8 air exchanges per hour. 

  Fans will be mounted on noise-absorbing plinth and equipped with a back-flow
  dumper. Spare air will be supplied through weather-resistant louvers, shut-off
  dumpers and G4 filters. Exhaust fans will be turned on based on temperature
  sensors placed in Machinery Room.

}

\twoColls{
  Dále jsou do strojovny navrženy další dva odsávací ventilátory pro odsávání
  zápachu při nakládání kalových briket do nákladního automobilu. Jeden nástěnný
  axiální ventilátor o vzduchovém výkonu 10.000 m3/h je osazen do fasády pod
  plošinou kalolisu a druhý střešní ventilátor o vzduchovém výkonu 10.000 m3/h je
  na střeše, půdorysně v místě kalolisu. Oba odsávací ventilátory budou spouštěny
  ručně vypínačem umístěným vedle vrat.

}{
  Machinery Room will be equipped with two supplemental exhaust fans to extract
  odor arising during loading sludge pallets into a truck.  First of these fans
  is wall axial fan providing air flow rate of 10,000 m3/h, which will be
  installed in facade below a platform of filter press.  The second fan will be
  installed on the roof above the filter press.  Both of these fans will be
  turned on manually by switch located next to the gate.

}

\twoColls{
  V prostoru strojovny jsou umístěny čtyři vany s teplotou vody 25\gc, o celkové
  vodní ploše 43.2 m2. Při vnitřní teplotě vzduchu +10\gc je předpoklad odparu vody
  cca 7 kg/h.  
  V zimním období stačí pro odvod vlhkostní zátěže vzduchový výkon
  7,000m3/h. Aby se strojovna nepodchlazovala, bude vzduchový výkon odsávacího
  ventilátoru řízen plynule od prostorových čidel relativní vlhkosti.

}{
  Four baths with a water of temperature of 25\gc are located in Machinery
  Room. Overall water area of these baths is 43.2 m2 causing evaporation of
  app. 7 kg of water per hour at indoor air temperature of 10\gc.
  During a winter year season, moist load will be exhausted by supplying of
  7,000 m3 of air per hour. To prevent Machinery Room over-cooling, air flow
  rate will be controlled continuously based on relative humidity room sensors.

}

\twoColls{
  Návrhové parametry ventilátorů, jsou uvedeny v příloze: TABLE EQUIPMENT AND
  PRINCIPLE DIAGRAMS.

}{
  Fans design parameters are listed in annex: TABLE EQUIPMENT AND PRINCIPLE
  DIAGRAMS.
}




\clearpage
\section{Equipment No.09.09-2 - Switch room}

\twoColls{
  Větrání a odvod tepelné zátěže z switch room, bude zajištěno kombinovaným
  systémem nuceného podtlakového odsávání a chlazení vzduchu pomocí cirkulační
  jednotky SPLIT. 
  V chladném období roku, bude místnost chlazena pouze venkovním
  vzduchem. Venkovní vzduch, bude do místnosti nasáván z fasády přes protidešťovou
  žaluzii, uzavírací klapku a filtr G4.  
  Vzduch z místnosti bude odsáván pomocí
  střešního ventilátoru osazeném na hluk tlumícím soklu se zpětnou klapkou.

}{
  Ventilation and exhaust of heat load from Switch Room will be ensured by a
  combined system of forced vacuum suction and air cooling by circulation 
  SPLIT unit. 
  During a cold year season, Switch Room will be cooled by outdoor air only,
  which will be supplied from building facade through weather resistant louver,
  shut-off dumper and G4 filter. 
  Indoor air will be exhausted by a roof fan mounted on noise-absorbing plinth
  and equipped with back-flow dumper.

}

\twoColls{
  Vzduchový výkon ventilátoru 1200 m3/h zajistí v místnosti 11x výměnu vzduchu za
  hodinu. 
  V teplém období roku, bude odsávací střešní ventilator vypnut a tepelná
  zátěž z místnosti bude odváděna pomocí cirkulační chladící jednotky SPLIT
  (jednotka SPLIT je předmětem projektu chlazení).

}{
  Air flow rate of 1.200 m3/h will ensure 11 air exchanges per hour.  
  During a warm year season, the exhaust roof fan will be turned off and heat
  load from Switch Room will be exhausted by a circulation cooling SPLIT unit
  (SPLIT unit is subject to cooling project).

}

\twoColls{
  Návrhové parametry ventilátoru, jsou uvedeny v příloze: TABLE EQUIPMENT AND
  PRINCIPLE DIAGRAMS.

}{
  Fan design parameters are listed in annex: TABLE EQUIPMENT AND PRINCIPLE
  DIAGRAMS.

}





\clearpage
\section{Equipment No.09.09-3 - Storage of spare parts}

\twoColls{
  Větrání skladu je řešeno pomocí jednoho střešního ventilátoru. Vzduchový výkon
  je navržen na 4x výměny vzduchu za hodinu. Ventilátor bude osazen na hluk
  tlumícím soklu a bude vybaven zpětnou klapkou. Náhradní vzduch bude přisáván
  z venkovního prostoru přes sestavu složenou  z protidešťové žaluzie, uzavírací
  klapky a filtru G4.

}{
  Ventilation of Storage of Spare Parts will be ensured by single roof fan. Air
  flow rate is designed to ensure 4 air exchanges per hour. 
  Fan will be mounted on noise-absorbing plinth and equipped with back-flow
  dumper. Spare air will be supplied from outdoor area through
  weather-resistant louver, shut-off dumper and G4 filter. 

}

\twoColls{
  Návrhové parametry ventilátoru, jsou uvedeny v příloze: TABLE EQUIPMENT AND
  PRINCIPLE DIAGRAMS.

}{
  Fan design parameters are listed in annex: TABLE EQUIPMENT AND PRINCIPLE
  DIAGRAMS.

}



\clearpage
\section{Equipment No.09.09-4 - Demi water station and pumping room}

\twoColls{
  Větrání místnosti a odvod tepelné zátěže je řešeno pomocí jednoho střešního
  ventilátoru. Vzduchový výkon ventilátoru 3000m3/h zajistí v místnosti 5x výměnu
  vzduchu za hodinu. Ventilátor bude osazen na hluk tlumícím soklu a bude vybaven
  zpětnou klapkou. Náhradní vzduch bude přisáván z venkovního prostoru přes
  sestavu složenou  z protidešťové žaluzie, uzavírací klapky a filtru G4.

}{
  Ventilation and exhaust of heat load from Demi Water Station and Pumping Room
  will be ensured by a single roof fan. Air flow rate of 3,000 m3/h is designed to
  ensure 5 air exchanges per hour. Fan will be mounted on noise-absorbing plinth
  and equipped with back-flow dumper. Spare air will be supplied from outdoor
  area through weather resistant louver, shut-off dumper and G4 filter.

}

\twoColls{
  Návrhové parametry ventilátoru, jsou uvedeny v příloze: TABLE EQUIPMENT AND
  PRINCIPLE DIAGRAMS.

}{
  Fan design parameters are listed in annex: TABLE EQUIPMENT AND PRINCIPLE
  DIAGRAMS.

}



\clearpage
\section{Equipment No.09.09-5 - Dosing of flacculant}

\twoColls{
  Větrání místnosti a odvod tepelné zátěže je řešeno pomocí jednoho střešního
  ventilátoru. Vzduchový výkon ventilátoru 1500m3/h zajistí v místnosti 4,4x
  výměnu vzduchu za hodinu. Ventilátor bude osazen na hluk tlumícím soklu a bude
  vybaven zpětnou klapkou. Náhradní vzduch bude přisáván z venkovního prostoru
  přes sestavu složenou  z protidešťové žaluzie, uzavírací klapky a filtru G4.

}{
  Ventilation and exhaust of heat load from Dosing of Flacculant
  will be ensured by a single roof fan. Air flow rate of 1,500 m3/h is designed to
  ensure 4.4 air exchanges per hour. Fan will be mounted on noise-absorbing plinth
  and equipped with back-flow dumper. Spare air will be supplied from outdoor
  area through weather resistant louver, shut-off dumper and G4 filter.

}

\twoColls{
  Návrhové parametry ventilátoru, jsou uvedeny v příloze: TABLE EQUIPMENT AND
  PRINCIPLE DIAGRAMS.

}{
  Fan design parameters are listed in annex: TABLE EQUIPMENT AND PRINCIPLE
  DIAGRAMS.

}



\clearpage
\section{Equipment No.09.09-6 - Ventilation}

\twoColls{
  Větrání a odvod tepelné zátěže z AC Machinery, bude zajištěno kombinovaným
  systémem nuceného podtlakového odsávání a chlazení vzduchu pomocí cirkulační
  jednotky SPLIT. V chladném období roku, bude místnost chlazena pouze venkovním
  vzduchem. Venkovní vzduch, bude do místnosti nasáván z fasády přes protidešťovou
  žaluzii, uzavírací klapku a filtr G4.  Vzduch z místnosti bude odsáván pomocí
  střešního ventilátoru osazeném na hluk tlumícím soklu se zpětnou klapkou.

}{
  Ventilation and exhaust of heat load from AC Machinery Room will be ensured by a
  combined system of forced vacuum suction and air cooling by circulation SPLIT
  unit. During a cold year season, AC Machinery Room will be cooled by outdoor air only, which
  will be supplied from building facade through weather-resistant louver,
  shut-off dumper and G4 filter. Indoor air will be exhausted by a roof fan
  mounted on noise-absorbing plinth equipped with back-flow dumper. 

}

\twoColls{
  Vzduchový výkon ventilátoru 1000m3/h zajistí v místnosti 3x výměnu vzduchu za
  hodinu. V teplém období roku, bude odsávací střešní ventilator vypnut a tepelná
  zátěž z místnosti bude odváděna pomocí cirkulační chladící jednotky SPLIT
  (jednotka SPLIT je předmětem projektu chlazení).

}{
  Air flow rate of 1,000 m3/h will ensure 3 air exchanges per hour.  During a
  warm year season, the exhaust roof fan will be turned off and heat load from
  AC Machinery Room will be exhausted by a circulation cooling SPLIT unit (SPLIT
  unit is subject to cooling project).

}

\twoColls{
  Návrhové parametry ventilátoru, jsou uvedeny v příloze: TABLE EQUIPMENT AND
  PRINCIPLE DIAGRAMS.

}{
  Fan design parameters are listed in annex: TABLE EQUIPMENT AND PRINCIPLE
  DIAGRAMS.

}


\clearpage
\section{Equipment No.09.09-7 - Laboratory}

\twoColls{
  Větrání místností control room a laboratory zajišťuje jedna vzduchotechnická
  rekuperační jednotka osazená ve strojovně vzduchotechniky. Teplotně upravovaný
  venkovní vzduch je přiváděn potrubím do vnitřních prostor pomocí vířivých
  anemostatů, jež jsou osazeny do podhledu, a znehodnocený vzduch je odsáván
  potrubím s vyústkami zaústěnými do podhledu. 
  V případě potřeby bude přiváděný vzduch dohříván v elektrickém ohřívači. 
  Přiváděná dávka venkovního vzduchu je
  min. 50m3/h na osobu, nebo min. 4x/h.

}{
  Ventilation of Control and Laboratory Rooms will be ensured by a single air
  handling recuperation unit placed in the AC Machinery Room. Thermally
  conditioned outdoor air will be forced through ventilation duct to indoor area
  by swirl diffusers mounted in suspended ceiling. Waste air will be exhausted
  by ducts equipped with outlets mounted into suspended ceiling. 
  If necessary, electrical heater will be used to thermally condition supplied
  air.  The amount of outdoor air supplied to breaking room will be 50~m3/h per
  person or at least 4 air exchanges per hour. 

}

\twoColls{
  Vzduchotechnická jednotka je ve vnitřním provedení, opatřena filtrem vzduchu
  (přívod F5+F7 /odtah F5), deskovým výměníkem s obtokem, elektrickým ohřívačem a
  dvěma ventilátory s frekvenčními měniči, nebo EC motory. 
  
  Zařízení pracuje s nuceným přívodem a odvodem vzduchu. Čerstvý vzduch je
  nasáván VZT jednotkou potrubím z fasády objektu, kde je umístěna protidešťová
  žaluzie. Znehodnocený vzduch je VZT jednotkou vyfukován nad střechu objektu. 

}{
  Innerly designed AHU will be equipped with air filters (F5+F7 supply / F5
  exhaust), rotatory heat exchanger with bypass, electrical heater and two fans
  with frequency changers or EC engines.

  Equipment will be working with forced air inlet and outlet. Fresh air will be
  suctioned in by AHU through ducts from building facade, where a weather
  resistant louver will be installed. Waste air will be blown by AHU over the
  roof of the building.

}

\twoColls{
  Chlazení místností zajišťuje profese chlazení pomocí dvou kazetových jednotek
  SPLIT.

}{
  Cooling of Control and Laboratory Rooms is ensured by profession of Cooling by
  two cassette SPLIT units.

}

\twoColls{
  Návrhové parametry jednotky, jsou uvedeny v příloze: TABLE EQUIPMENT AND
  PRINCIPLE DIAGRAMS.

}{
  Unit design parameters are listed in annex: TABLE EQUIPMENT AND PRINCIPLE
  DIAGRAMS.

}




\clearpage
\section{Equipment No.09.09-8 - Storage of chemicals}

\twoColls{
  Větrání skladu je řešeno pomocí jednoho střešního ventilátoru. Vzduchový výkon
  ventilátoru 1500m3/h zajistí v místnosti 6x výměnu vzduchu za hodinu. Ventilátor
  bude osazen na hluk tlumícím soklu a bude vybaven zpětnou klapkou. Náhradní
  vzduch bude přisáván z venkovního prostoru přes sestavu složenou  z protidešťové
  žaluzie, uzavírací klapky a filtru G4.

}{
  Ventilation of Storage of Chemicals will be ensured by single roof fan. Air
  flow rate of 1,500 m3/h is designed to ensure 6 air exchanges per hour. 
  Fan will be mounted on noise-absorbing plinth and equipped with back-flow
  dumper. Spare air will be supplied from outdoor area through
  weather-resistant louver, shut-off dumper and G4 filter. 
}

\twoColls{
  Návrhové parametry ventilátoru, jsou uvedeny v příloze: TABLE EQUIPMENT AND
  PRINCIPLE DIAGRAMS.

}{
  Fan design parameters are listed in annex: TABLE EQUIPMENT AND PRINCIPLE
  DIAGRAMS.

}



\clearpage
\section{Equipment No.09.09-9 - Machine room of dosing}

\twoColls{
  Větrání místnosti, kde dochází k dávkování dioxinu chlóru, je řešeno nuceným
  podtlakovým způsobem, pomocí radiálního ventilátoru. 
  Vzduchový výkon havarijního větrání je navržen na 10x výměn vzduchu za hodinu.
  Ventilátor z PVC s odolností pro odsávání plynného chlóru, bude s motorem mimo
  proud vzdušiny, osazen na střeše objektu. 
  Odsávací potrubí bude v místnosti staženo až k podlaze a odsávání
  znehodnoceného vzduchu bude zajištěno pomocí tří vyústek umístěných u podlahy,
  v polovině výšky a pod stropem. 

}{
  Ventilation of Machine Room of Dosing, where chlorine dioxin will be being dosed,
  will be ensured by a vacuum way using radial fan. 
  Air flow rate of emergency ventilation will ensure 10 air exchanges per hour.
  PVC fan, which is resistant to chlorine gas, and its engine will be mounted
  apart from air flow on the roof of the building.
  Exhaust duct in the room will be led down to the floor. Waste air will be
  exhausted through three outlets placed below the suspended ceiling in the
  half of room height. 


}

\twoColls{
  Náhradní vzduch bude do místnosti nasáván přes stěnové mřížky z machinery room. 

}{
  Spare air will be supplied through wall grilles from Machinery Room.

}

\twoColls{
  Návrhové parametry ventilátoru, jsou uvedeny v příloze: TABLE EQUIPMENT AND
  PRINCIPLE DIAGRAMS.

}{
  Fan design parameters are listed in annex: TABLE EQUIPMENT AND PRINCIPLE
  DIAGRAMS.

}
