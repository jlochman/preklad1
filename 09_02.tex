\section{Equipment No.09.02-1	- Production area}

\twoColls{
  Popis tohoto zařízení je shodný jako Equipment No.09.01-1 - Production area.

}{
  Description of this equipment coincides with that of Equipment No.09.01-1 -
  Production area.

}

\twoColls{
  Návrhové parametry jednotky, jsou uvedeny v příloze: TABLE EQUIPMENT AND
  PRINCIPLE DIAGRAMS.

}{
  Unit design parameters are listed in annex: TABLE EQUIPMENT AND PRINCIPLE DIA-
  GRAMS.

}

\clearpage
\section{Equipment No.09.02-2	- Raw material room}

\twoColls{
  Větrání, chlazení a vytápění Raw material room zajišťuje jedna vzduchotechnická
  jednotka osazená ve strojovně vzduchotechniky v 2NP.  Teplotně upravovaný
  venkovní vzduch je přiváděn potrubím do prostoru, kam je vyfukován pomocí
  vyústek. Přívodní vyústky jsou ovládány servomotory na 24V, dle prostorvé
  teploty.  Odvod vzduchu je pomocí odsávacího potrubí, vedeného pod stropem
  větraného prostoru.

}{
  Ventilation, cooling and heating of Raw material room will be ensured by
  single AHU placed in AC machinery room on the 2nd floor. Thermally conditioned
  outdoor air will be forced through ventilation duct to raw material room,
  where it will be blown out through grills. Supply air diffusers will be
  controlled by 24 V servomotors according to temperature in the room. Waste air
  will be suctioned out by ducts leading below the ceiling.

}

\twoColls{
  Množství přiváděného vzduchu je navrženo na 4 výměny vzduchu za hodinu do výšky
  místnosti 3m. 

}{
  The amount of supplied air is designed to ensure 4 air exchanges per hour up
  to room height of 3 m.

}

\twoColls{
  Vzduchotechnická jednotka je ve vnitřním provedení, opatřena filtrem vzduchu
  (přívod F5+F7 / odtah F5), rotačním výměníkem, směšovací komorou, vodním
  chladičem (15 / 20\gc), vodním ohřívačem (45 / 30\gc) a dvěma ventilátory
  s frekvenčními měniči. Zařízení pracuje s nuceným přívodem a odvodem vzduchu.
  Čerstvý vzduch je nasáván VZT jednotkou potrubím z fasády objektu, kde je
  umístěna protidešťová žaluzie. Od jednotky je vzduch veden VZT potrubím pod
  stropem větraného prostoru. Znehodnocený vzduch je VZT jednotkou vyfukován nad
  střechu objektu.       

}{
  Innerly designed AHU will be equipped with air filters (F5+F7 supply / F5 exhaust), 
  rotatory heat exchanger, mixing chamber, water-based air cooler
  (15 / 20\gc), water-based air heater (45 / 30\gc) and two fans with frequency
  changers. Equipment will work with forced air inlet and outlet. Fresh air will
  be suctioned in by AHU through ducts from building facade, where a weather
  resistant louver will be installed. Air from AHU will be led by ventilation ducts
  below the ceiling of ventilated area. Waste air will be blown out by AHU over
  the roof of the building.

}

\twoColls{
  Návrhové parametry jednotky, jsou uvedeny v příloze: TABLE EQUIPMENT AND
  PRINCIPLE DIAGRAMS.

}{
  Unit design parameters are listed in annex: TABLE EQUIPMENT AND PRINCIPLE
  DIAGRAMS.

}

\clearpage
\section{Equipment No.09.02-3	- AC machinery room}

\twoColls{
  Větrání strojovny vzduchotechniky je řešeno pomocí axiálního potrubního
  ventilátoru. Vzduchový výkon je navržen na 3x výměnu vzduchu za hodinu. U
  axiálního ventilátoru je do potrubí osazena klapka a tlumič hluku. Náhradní
  vzduch bude přisáván z venkovního prostoru přes protidešťovou žaluzii, uzavírací
  klapku a filtr G4.

}{
  Ventilation of AC machinery room will be ensured by an axial duct fan. Air flow rate
  is designed for 3 air exchanges per hour. Axial fan will be equipped
  with damper and silencer. Spare are will be suctioned in through
  weather resistant louver, shut-off damper and G4 filter.

}

\twoColls{
  Návrhové parametry ventilátoru, jsou uvedeny v příloze: TABLE EQUIPMENT AND
  PRINCIPLE DIAGRAMS.

}{
  Fan design parameters are listed in annex: TABLE EQUIPMENT AND PRINCIPLE
  DIAGRAMS.

}

