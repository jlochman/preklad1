\section{Equipment No.09.04-1	- Quality lab}

\twoColls{
  Větrání, chlazení, vytápění, vlhčení a odvlhčování prostoru Quality lab
  zajišťuje soustava dvou vzduchotechnických jednotek a jedné adsorpční
  odvlhčovací jednotky. 

  Všechny jednotky jsou osazeny ve strojovně vzduchotechniky
  v 2NP. Jednotka MAU a adsorpční jednotka slouží pro úpravu venkovního vzduchu o
  objemu 30\% z celkového množství vzduchu přiváděného do Quality lab. 

  Jednotka AHU je pouze cirkulační a přivádí teplotně a vlhkostně upravovaný vzduch potrubím do
  klimatizovaných prostor, kam je vyfukován pomocí vířivých anemostatů  osazených
  do podhledu. 

  Odvod vzduchu je přes odsávací potrubí vyústěného nad podhledem.
  Propojení místnosti s odsávacím potrubím je zajištěno pomocí mřížovaných kazet
  osazených do podhledu. Množství přiváděného vzduchu je navrženo v místnosti
  Quality lab, dle potřeby odvodu tepelné zátěže a v ostatních prostorách,   QL
  sampler room , QL meeting room a QL office dle standardu LEGO.	

  Vzduchotechnická jednotka AHU, je ve vnitřním provedení, opatřena filtrem vzduchu F7, vodním
  chladičem (9 / 15\gc), vodním ohřívačem (45 / 30\gc), přívodním ventilátorem
  s frekvenčním měničem a zvlhčovací komorou. Do komory je osazena parní trubice,
  napojená na lokální elektrický odporový vyvíječ páry.

}{
  Ventilation, cooling, heating, humidification and dehumidification of Quality
  lab will be ensured by two AHUs and one adsorption dehumidifying unit.

  All units will be placed in AC machinery room on the 2nd floor. MAU and
  adsorption unit will condition 30~\% of total amount of air supplied to Quality
  lab.

  AHU will be circulation only and will supply thermally conditioned and humidity
  regulated air through a duct to climatized areas, where it will be blown by swirl
  diffusers mounted in suspended ceiling. 

  Waste air will be exhausted through an exhaust duct above the suspended ceiling. Rooms
  will be interconnected with exhaust duct by reticulated cartridges mounted in
  the suspended ceiling. Amount of supplied air will be set in the Quality lab room in
  order to lead away heat load to comply with LEGO Standards in the
  following rooms: Sampler room, QL meeting room and QL office.

  Innerly designed AHU will be equipped with F7 air filter, water-based air cooler
  (9 / 15\gc), water-based air heater (45 / 30\gc), supply fan with frequency
  changer and humidification chamber. This chamber will be equipped with steam pipe
  connected to a local electrical resistance-based steam producer.  
}

\twoColls{
  Vzduchotechnická jednotka MAU, je ve vnitřním provedení, opatřena filtrem
  vzduchu (přívod F5 / odtah F5), rotačním výměníkem, vodním chladičem (9 / 15\gc),
  vodním ohřívačem (45 / 30\gc) a dvěma ventilátory s frekvenčními měniči.

}{
  Innerly designed MAU will be equipped with air filters (F5 suppy / F5
  exhaust), rotatory heat exchanger, water-based air coller (9 / 15\gc),
  water-base air heater (45 / 30\gc) and two fans with frequency changers. 
}

\twoColls{
  Vzduchotechnická adsorpční jednotka, je ve vnitřním provedení, opatřena filtrem
  vzduchu (přívod F5 / odtah F5), rotačním adsorpčním výměníkem, elektrickým
  ohřívačem, a dvěma ventilátory s fre\-kven\-ční\-mi měniči.

}{
  Innerly designed air conditioning adsorption unit will be equipped with air
  filters (F5 supply / F5 exhaust), rotatory adsorption exchanger, electrical air
  heater and two fans with frequency changers. 

}

\twoColls{
  V dalším stupni dokumentace, je uvažováno se sloučením MAU jednotky a adsorpční
  jednotky do jedné jednotky od jednoho výrobce.

}{
  In next stage of documentation, MAU and adsorption unit may be fused into one
  unit from one manufacturer. 

}

\twoColls{
  Jelikož jsou jedním centrálním zařízením klimatizovány čtyři prostory, jsou
  v přívodním potrubí osazeny regulátory průtoku VAV s vodními ohřívači a
  v odvodním potrubí regulátory průtoku VAV. 

  Regulátory průtoku VAV s ohřívači na
  přívodu budou regulovány dle prostorového teplotního čidla. Regulátory VAV na
  odvodu budou regulovány souběžně s regulátory VAV na přívodu. 

  Jelikož je v místnosti Quality lab lokální odtah od technologie, bude VAV regulátor na
  odtahu ovládán na základě konstantního přetlaku cca 10 až 20 Pa v místnosti
  Quality lab., popřípadě dle chodu odsávacích ventilátorů technologie. 

  Otáčky ventilátorů v centrálních jednotkách budou řízeny na konstantní tlak v potrubí.
  Parní výkon zvlhčování bude řízen dle prostorového čidla relativní vlhkosti
  v místnosti Quality lab.

}{
  Since 4 areas will be climatized by one central equipment, inlets will be
  equipped with flow rate VAV controllers with water-based air heaters, whereas
  outlets will be equipped with flow rate VAV controllers only.

  Whole equipment in inlets will be controlled according to temperature room sensor.
  As soon as these controllers are set, controller settings on outlets will be
  determined. 

  In Quality lab room, there will be a minimal heat load from technology, which
  must be ventilated. This will be done by ensuring overpressure of 10-20~Pa by
  VAV controller in outlet, which will depend on operation of exhaust fans of
  technology itself.  

  Fan speeds in central AHUs will be set to ensure constant pressure in supply and
  exhaust ducts. Power of steam humidification will be controlled by relative
  humidity sensor placed in Quality lab room.

}

\twoColls{
  Návrhové parametry jednotek a ventilátorů, jsou uvedeny v příloze: TABLE
  EQUIPMENT AND PRINCIPLE DIAGRAMS.

}{
  Units and fans design parameters are listed in annex: TABLE EQUIPMENT AND
  PRINCIPLE DIAGRAMS.

}

\clearpage
\section{Equipment No.09.04-2	- P-box Washing}

\twoColls{
  Větrání, chlazení a vytápění P-box washing zajišťuje jedna vzduchotechnická
  jednotka osazená ve strojovně vzduchotechniky v 2NP.  Teplotně upravovaný
  venkovní vzduch je přiváděn potrubím do prostoru, kam je vyfukován pomocí
  vyústek. Přívodní vyústky jsou ovládány servomotory na 24V, dle prostorvé
  teploty.  Odvod vzduchu je pomocí odsávacího potrubí, vedeného pod stropem
  větraného prostoru. 

  Jelikož je v místnosti technologický odtah bude na přívodním
  potrubí osazen regulátor konstantního průtoku CAV a na odvodním potrubí
  regulátor proměnného průtoku VAV. Regulátor VAV na odtahu bude ovládán na
  základě konstantního podtlaku max. 50Pa v místnosti, popřípadě dle chodu
  odsávacích ventilátorů technologie. Otáčky ventilátorů v centrální jednotce
  budou řízeny na konstantní tlak v potrubí. 

}{
  Ventilation, cooling and heating of P-box Washing will be ensured by single
  AHU placed in AC machinery room on the 2nd floor.  Thermally conditioned
  outdoor air will be supplied through ventilation duct to indoor areas by
  diffusers. Inlet diffusers will be controlled by 24~V servomotors according to
  room temperature. Waste air will be suctioned out through ducts leading below the
  ceiling of ventilated area. 

  Since there will be a heat load from technology, inlet duct will
  be equipped with CAV controller ensuring constant air flow rate. Outlet duct
  will be equipped with VAV controller allowing varying air flow rate. This
  controller will be set in order to ensure constant underpressure in room, max.
  50~Pa, optionally depending on operation of exhaust fans of technology. Fan speeds in
  central unit will be controlled to ensure constant pressure in duct. 
}

\twoColls{
  Množství přiváděného vzduchu je navrženo na 4 výměny vzduchu za hodinu do výšky
  místnosti 3m. 

}{
  The amount of supplied air is designed to ensure 4 air exchanges per hour up
  to room height of 3 m.

}

\twoColls{
  Vzduchotechnická jednotka je ve vnitřním provedení, opatřena filtrem vzduchu
  (přívod F5+F7 / odtah F5), deskovým výměníkem s obtokem, směšovací komorou,
  vodním chladičem (15/20\gc), vodním chladičem (9/15\gc), vodním ohřívačem
  (45 / 30\gc) a dvěma ventilátory s frekvenčními měniči. 

  Zařízení pracuje s nuceným
  přívodem a odvodem vzduchu. Čerstvý vzduch je nasáván VZT jednotkou potrubím
  z fasády objektu, kde je umístěna protidešťová žaluzie. Znehodnocený vzduch je
  VZT jednotkou vyfukován nad střechu objektu.       

}{
  Innerly designed AHU will be equipped with air filters (F5+F7 supply / F5 exhaust),
  plate heat exchanger with bypass, mixing chamber, water-based air cooler (15 /
  20\gc), water-based air cooler (9 / 15\gc) water-based air heater
  (45 / 30\gc) and two fans with frequency changers. 

  Equipment will work with forced air inlet and outlet. Fresh air will be
  suctioned in by AHU through ducts from building facade, where a weather
  resistant louver will be placed.  Waste air will be blown by AHU over the roof
  of the building.

}

\twoColls{
  Cirkulační klapka ve směšovací komoře bude v běžném provozu trvale zavřená a
  jednotka bude pracovat pouze s čerstvým vzduchem. Při odstávce technologie
  v zimním období, pokud bude třeba vytápět vnitřní prostor, budou klapky ve
  směšovací komoře přepnuty do 100\% cirkulace a odsávací ventilátor bude vypnut. 

}{
  During standard operation, circulation valve in mixing chamber will be closed
  and the unit will work in mode with fresh air only. During a technology
  shutdown in winter season, when there will be a necessity to heat indoor
  areas, valves in mixing chamber will be switched to 100~\% circulation mode
  and exhaust fan will be turned off.

}

\twoColls{
  Návrhové parametry jednotky, jsou uvedeny v příloze: TABLE EQUIPMENT AND
  PRINCIPLE DIAGRAMS.

}{
  Unit design parameters are listed in annex: TABLE EQUIPMENT AND PRINCIPLE
  DIAGRAMS.

}

\clearpage
\section{Equipment No.09.04-3	- Mould Bunker}

\twoColls{
  Větrání, chlazení a vytápění Mould bunker zajišťuje jedna vzduchotechnická
  jednotka osazená ve strojovně vzduchotechniky v 2NP. Teplotně upravovaný
  venkovní vzduch je přiváděn potrubím do prostoru, kam je vyfukován pomocí
  vířivých vyústek. Odvod vzduchu je pomocí odsávacího potrubí místně zústěného do
  větraného prostoru.

}{
  Ventilation, cooling and heating of Mould Bunker will be ensured by single AHU
  placed in AC machinery room on the 2nd floor. Thermally conditioned outdoor air
  will be forced through ventilation duct to equipment room, where it will be
  blown out through swirl diffusers. Waste air will be suctioned out through ducts
  locally connected to ventilated space.

}

\twoColls{
  Množství přiváděného vzduchu je navrženo na 4 výměny vzduchu za hodinu do výšky
  místnosti 3m. 

}{
  The amount of supplied air is designed to ensure 4 air exchanges per hour up
  to room height of 3 m.

}

\twoColls{
  Vzduchotechnická jednotka je ve vnitřním provedení, opatřena filtrem vzduchu
  (přívod F5+F7 / odtah F5), rotačním výměníkem, směšovací komorou, vodním
  chladičem (15/20\gc), vodním chla\-di\-čem (9/15\gc), vodním ohřívačem (45/30\gc) a
  dvěma ventilátory s frekvenčními měniči. Zařízení pracuje s nuceným přívodem a
  odvodem vzduchu. Čerstvý vzduch je nasáván VZT jednotkou potrubím z fasády
  objektu, kde je umístěna protidešťová žaluzie. Znehodnocený vzduch je VZT
  jednotkou vyfukován nad střechu objektu.       

}{ 
  Innerly designed AHU will be equipped with air filters (F5+F7 supply / F5
  exhaust), rotatory heat exchanger, mixing chamber, water-based air cooler (15
  / 20\gc), water-based air cooler (9 / 15\gc), water-based air heater (45 /
  30\gc) and two fans with frequency changers.  Equipment will work with forced
  air inlet and outlet. Fresh air will be suctioned in by AHU through ducts from
  building facade, where a weather resistant louver will be installed. Waste air
  will be blown out by AHU over the roof of the building.  

}

\twoColls{
  Do přívodního a odsávacího potrubí musí být osazeny z vnitřního a vnějšího líce
  stěny (požárního předělu) Mould bunkru, za sebou dvě požární klapky, kouřotěsné
  s požární odolností  180minut. 

}{
  Supply and exhaust ducts will have to be equipped on both, inner and outer, faces of
  Mould Bunker wall with two consecutive fire dampers, which will be smoke proof
  and will have fire resistance of 180 minutes.

}

\twoColls{
  Návrhové parametry jednotky, jsou uvedeny v příloze: TABLE EQUIPMENT AND
  PRINCIPLE DIAGRAMS.

}{
  Unit design parameters are listed in annex: TABLE EQUIPMENT AND PRINCIPLE
  DIAGRAMS.

}

\clearpage
\section{Equipment No.09.04-4	- Mould Maintenance}

\twoColls{
  Větrání, chlazení a vytápění prostor Mould Maintenance zajišťují dvě
  vzduchotechnické jednotky osazené ve strojovně vzduchotechniky v 2NP. Jednotka
  poz. 4.1 zajišťuje větrání velkého prostoru Molud Maintenance  a prostoru
  Ultraconic clearing room. Jednotka poz. 4.2 zajišťuje větrání menších místností
  náležících k Moulding Maintenance a Machine Equipments Maintenance.

}{
  Ventilation, cooling and heating of Mould Maintenance will be ensured by two
  AHUs placed in AC machinery room on the 2nd floor. Unit ref. 4.1 will
  ensure ventilation of overall area of Mould Maintenance and Ultraconic
  Clearing Room. Unit ref. 4.2 will ensure ventilation of smaller rooms
  belonging to Moulding Maintenance and Machine Equipments Maintenance.

}

\twoColls{ 
  Jednotkou poz. 4.1 je teplotně upravovaný venkovní vzduch přiváděn
  potrubím do prostoru, kam je vyfukován pomocí vyústek. Přívodní vyústky
  umístěné v Moulding Maintenance jsou ovládány servomotory na 24V, dle
  prostorvé teploty.  

  Odvod vzduchu je pomocí odsávacího potrubí, vedeného pod
  stropem větraného prostoru. Jelikož je jedním zařízení větráno více prostor a
  dále jsou zde umístěny technologické odtahy budou na přívodním a odvodním
  potrubí osazeny regulátory proměnného průtoku VAV. 

}{
  Unit ref. 4.1 will supply thermally conditioned outdoor air through a duct to
  area, where it will be blown out by diffusers. Inlet diffusers placed in
  Moulding Maintenance will be controlled by 24~V servomotors according to room
  temperature. 

  Waste air will be exhausted by exhaust duct leading below the ceiling of ventilated
  area. Since many areas will be ventilated by one equipment and because of
  exhaust of heat load from technology, supply and exhaust ducts will be
  equipped with VAV controllers for varying air flow rate.
}

\twoColls{
  Regulátor průtoku VAV Ultraconic clearing room bude regulován dle prostorového
  teplotního čidla. Regulátory VAV na odvodu budou regulovány souběžně
  s regulátory VAV na přívodu. 

}{
  Flow rate VAV controller in inlet for Ultraconic clearing room will be set according to
  temperature room sensor. VAV controller in outlet will be set simultaneously
  with VAV controllers in inlet.

}

\twoColls{
  Jelikož je v místnostech lokální odtah od technologie, bude VAV regulátor na
  odtahu ovládán ještě na základě konstantního podtlaku max. 50Pa v místnosti,
  popřípadě dle chodu odsávacích ventilátorů technologie. Otáčky ventilátorů
  v centrální jednotce budou řízeny na konstantní tlak v potrubí. 

}{
  Since there will be a constant heat load from technology, VAV controller in
  outlet will have to ensure constant underpressure of max. 50~Pa and optionally
  will depend on operation of exhaust fans of technology. Fan speeds in central
  unit will be controlled to ensure constant pressure in duct.

}

\twoColls{
  Množství přiváděného vzduchu je navrženo dle standardu LEGO.  

}{
  The amount of supplied air is designed to comply with LEGO standard.

}

\twoColls{
  Vzduchotechnická jednotka je ve vnitřním provedení, opatřena filtrem vzduchu
  (přívod F5+F7 /odtah F5), rotačním výměníkem, směšovací komorou, vodním
  chladičem (15/20\gc), vodním chladičem (9/15\gc), vodním ohřívačem (45/30\gc) a
  dvěma ventilátory s frekvenčními měniči. 

  Zařízení pracuje s nuceným přívodem a
  odvodem vzduchu. 
  Čerstvý vzduch je nasáván VZT jednotkou potrubím z fasády
  objektu, kde je umístěna protidešťová žaluzie. Znehodnocený vzduch je VZT
  jednotkou vyfukován nad střechu objektu.       

}{
  Innerly designed AHU will be equipped with air filters (F5+F7 supply / F5
  exhaust), rotatory heat exchanger, mixing chamber, water-based air cooler (15
  / 20\gc), water-based air cooler (9 / 15\gc) water-based air heater (45 /
  30\gc) and two fans with frequency changers. 

  Equipment will work with forced air inlet and outlet.  Fresh air will be
  suctioned in by AHU through ducts from building facade, where a weather
  resistant louver will be installed. Waste air will be blown by AHU over the roof
  of the building.

}

\twoColls{
  Cirkulační klapka ve směšovací komoře bude v běžném provozu trvale zavřená a
  jednotka bude pracovat pouze s čerstvým vzduchem. Při odstávce technologie
  v zimním období, pokud bude třeba vytápět vnitřní prostor, budou klapky ve
  směšovací komoře přepnuty do 100\% cirkulace a odsávací ventilátor bude vypnut. 

}{
  During standard operation, circulation valve in mixing chamber will be closed
  and the unit will work in mode with fresh air only. During a technology
  shutdown in winter season, when there will be a necessity to heat indoor
  areas, valves in mixing chamber will be switched to 100~\% circulation mode
  and exhaust fan will be turned off.

}

\twoColls{
  Návrhové parametry jednotky, jsou uvedeny v příloze: TABLE EQUIPMENT AND
  PRINCIPLE DIAGRAMS.

}{
  Unit design parameters are listed in annex: TABLE EQUIPMENT AND PRINCIPLE
  DIAGRAMS.

}

\twoColls{
  Jednotkou poz. 4.2 jsou větrány místnosti MLD SHIPPING REPARATION, POLISH ROOM,
  LASER WELDING ROOM, HK STORAGE ROOM, ELECTRONIC WORKSHOP, SANDBOILER ROOM,
  WELDING ROOM, OFFICE a MEETING ROOM.   

}{
  Unit ref. 4.2 will ensure ventilation of following rooms: MLD SHIPPING
  REPARATION, POLISH ROOM, LASER WELDING ROOM, HK STORAGE ROOM, ELECTRONIC
  WORKSHOP, SANDBOILER ROOM, WELDING ROOM, OFFICE a MEETING ROOM.   

}

\twoColls{
  Teplotně upravovaný venkovní vzduch je přiváděn potrubím do vnitřních prostor
  pomocí vířivých anemostatů, jež jsou osazeny do podhledu, nebo nad technologii.
  Znehodnocený vzduch je odsáván potrubím vedeným nad podhledem, nebo nad
  technologii. Propojení mezi větranou místností a odsávaným podhledem, bude
  zajištěno mřížovanou podhledovou kazetou. 

}{
  Thermally conditioned outdoor air will be forced through ventilation duct to
  indoor areas, where it will be blown out through swirl diffusers mounted in
  suspended ceiling or above the technology.  Waste air will be exhausted through
  an exhaust duct below the ceiling or above the technology. Room will be
  interconnected with ventilation ducts by reticulated cartridges.

}

\twoColls{
  Jelikož, je jedním centrálním zařízením klimatizováno více prostor, jsou
  v přívodním potrubí osazeny regulátory průtoku VAV s vodními ohřívači a
  v odvodním potrubí regulátory průtoku VAV. Regulátory průtoku VAV s ohřívači na
  přívodu budou regulovány dle prostorového teplotního čidla z každé místnosti.
  Regulátory VAV na odvodu budou regulovány souběžně s regulátory VAV na přívodu.

  V místnostech, kde jsou lokální odtahy od technologie, bude VAV regulátor na
  odtahu ovládán na základě konstantního podtlaku v místnosti max. 50Pa, popřípadě
  dle chodu odsávacích ventilátorů technologie. Otáčky ventilátorů v centrálních
  jednotkách budou řízeny na konstantní tlak v potrubí. Množství přiváděného
  vzduchu je navrženo dle standardu LEGO.  

}{
  Since many areas will be climatized by one central equipment, inlets will be
  equipped with flow rate VAV controllers with water-based air heaters whereas
  outlets will be equipped with flow rate VAV controllers only. Whole equipment
  in inlets will be controlled according to temperature room sensor.  As soon as
  these controllers are set, controller settings on outlets will be determined. 

  In rooms with a local exhausts from technology, VAV controller in outlet will
  be set to ensure constant underpressure in room, max. 50~Pa, which optionally
  will depend on operation of exhaust fans of technology. Fan speeds in central
  units will be controlled to ensure constant pressure in ducts. 
  The amount of supplied air is designed to comply with LEGO standard.

}

\twoColls{
  Vzduchotechnická jednotka je ve vnitřním provedení, opatřena filtrem vzduchu
  (přívod F5+F7 /odtah F5), rotačním výměníkem, směšovací komorou, vodním
  chladičem (15/20\gc), vodním chladičem (9 / 15\gc), vodním ohřívačem (45/30\gc) a
  dvěma ventilátory s frekvenčními měniči. 
  
  Zařízení pracuje s nuceným přívodem a
  odvodem vzduchu. Čerstvý vzduch je nasáván VZT jednotkou potrubím z fasády
  objektu, kde je umístěna protidešťová žaluzie. Znehodnocený vzduch je VZT
  jednotkou vyfukován nad střechu objektu.       

}{
  Innerly designed AHU will be equipped with air filters (F5+F7 supply / F5
  exhaust), rotatory heat exchanger, mixing chamber, water-based air cooler
  (15 / 20\gc), water-based air cooler (9 / 15\gc) water-based air heater
  
  Equipment will work with forced air inlet and outlet. Fresh air will be
  suctioned in by AHU through ducts from building facade, where a weather
  resistant louver will be installed.  Waste air will be blown by AHU over the
  roof of the building.

}

\twoColls{
  Cirkulační klapka ve směšovací komoře bude v běžném provozu trvale zavřená a
  jednotka bude pracovat pouze s čerstvým vzduchem. Při odstávce technologie
  v zimním období, pokud bude třeba vytápět vnitřní prostor, budou klapky ve
  směšovací komoře přepnuty do 100\% cirkulace a odsávací ventilátor bude vypnut. 

}{
  During standard operation, circulation valve in mixing chamber will be closed
  and the unit will work in mode with fresh air only. During a technology
  shutdown in winter season, when there will be a necessity to heat indoor
  areas, valves in mixing chamber will be switched to 100~\% circulation mode
  and exhaust fan will be turned off.

}

\twoColls{
  Návrhové parametry jednotky, jsou uvedeny v příloze: TABLE EQUIPMENT AND
  PRINCIPLE DIAGRAMS.

}{
  Unit design parameters are listed in annex: TABLE EQUIPMENT AND PRINCIPLE
  DIAGRAMS.

}

\clearpage
\section{Equipment No.09.04-5	- Machine Equipments Maintenance}

\twoColls{
  Větrání, chlazení a vytápění Machine Equipments Maintenance a Spare Part Storage
  Room zajišťuje jedna vzduchotechnická jednotka osazená ve strojovně
  vzduchotechniky v 2NP.  Teplotně upravovaný venkovní vzduch je přiváděn potrubím
  do prostoru, kam je vyfukován pomocí vyústek. Přívodní vyústky jsou ovládány
  servomotory na 24V, dle prostorvé teploty z každé místnosti.  Odvod vzduchu je
  pomocí odsávacího potrubí, vedeného pod stropem větraného prostoru. 

}{
  Ventilation, cooling and heating of Machine Equipments Maintenance and Spare
  Part Storage Room will be ensured by single AHU placed in AC machinery room on
  the 2nd floor.  Thermally conditioned outdoor air will be supplied through
  ventilation duct to indoor areas by diffusers. Inlet diffusers will be
  controlled by 24~V servomotors according to temperature in each room. Waste
  air will be suctioned out through ducts leading below the ceiling of
  ventilated area. 

}

\twoColls{
  Jelikož, je jedním centrálním zařízením větráno více prostor, jsou v přívodním i
  odvodním potrubí  osazeny regulátory průtoku VAV. Regulátory průtoku VAV  na
  přívodu budou regulovány dle prostorového teplotního čidla z každé místnosti.
  Regulátory VAV na odvodu budou regulovány souběžně s regulátory VAV na přívodu.
  Otáčky ventilátorů v centrální jednotce budou řízeny na konstantní tlak
  v potrubí. 

}{
  Since many areas will be ventilated by one central equipment, inlets and
  outlets will be equipped with flow rate VAV controllers. VAV controllers in
  inlets will be set according to temperature sensor in each room. As soon as
  these controllers are set, controller settings on outlets will be determined.  Fan
  speeds in central AHUs will be set to ensure constant pressure in supply and
  exhaust ducts.

}

\twoColls{
  Množství přiváděného vzduchu je navrženo na 4 výměny vzduchu za hodinu do výšky
  místnosti 3m. 

}{
  The amount of supplied air is designed to ensure 4 air exchanges per hour up
  to room height of 3 m.

}

\twoColls{
  Vzduchotechnická jednotka je ve vnitřním provedení, opatřena filtrem vzduchu
  (přívod F5+F7 /odtah F5), rotačním výměníkem, směšovací komorou, vodním
  chladičem (15/20\gc), vodním chladičem (9/15\gc), vodním ohřívačem (45/30\gc) a
  dvěma ventilátory s frekvenčními měniči. 

  Zařízení pracuje s nuceným přívodem a
  odvodem vzduchu. Čerstvý vzduch je nasáván VZT jednotkou potrubím z fasády
  objektu, kde je umístěna protidešťová žaluzie. Znehodnocený vzduch je VZT
  jednotkou vyfukován nad střechu objektu.       

}{
  Innerly designed AHU will be equipped with air filters (F5+F7 supply / F5 exhaust),
  rotatory heat exchanger, mixing chamber, water-based air cooler (15 /
  20\gc), water-based air cooler (9 / 15\gc) water-based air heater
  (45 / 30\gc) and two fans with frequency changers. 

  Equipment will work with forced air inlet and outlet. Fresh air will be
  suctioned in by AHU through ducts from building facade, where a weather
  resistant louver will be placed.  Waste air will be blown by AHU over the roof
  of the building.

}

\twoColls{
  Cirkulační klapka ve směšovací komoře bude v běžném provozu trvale zavřená a
  jednotka bude pracovat pouze s čerstvým vzduchem. Při odstávce technologie
  v zimním období, pokud bude třeba vytápět vnitřní prostor, budou klapky ve
  směšovací komoře přepnuty do 100\% cirkulace a odsávací ventilátor bude vypnut. 

}{
  During standard operation, circulation valve in mixing chamber will be closed
  and the unit will work in mode with fresh air only. During a technology
  shutdown in winter season, when there will be a necessity to heat indoor
  areas, valves in mixing chamber will be switched to 100~\% circulation mode
  and exhaust fan will be turned off.

}

\twoColls{
  Návrhové parametry jednotky, jsou uvedeny v příloze: TABLE EQUIPMENT AND
  PRINCIPLE DIAGRAMS.

}{
  Unit design parameters are listed in annex: TABLE EQUIPMENT AND PRINCIPLE
  DIAGRAMS.

}

\clearpage
\section{Equipment No.09.04-6	- Central Grinding}

\twoColls{
  Větrání, chlazení a vytápění Central Grinding zajišťuje jedna vzduchotechnická
  jednotka osazená ve strojovně vzduchotechniky v 2NP.  Teplotně upravovaný
  venkovní vzduch je přiváděn potrubím do prostoru, kam je vyfukován pomocí
  vyústek. Přívodní vyústky jsou ovládány servomotory na 24V, dle prostorvé
  teploty.  Odvod vzduchu je pomocí odsávacího potrubí, vedeného pod stropem
  větraného prostoru. 

  Jelikož je v místnosti technologický odtah bude na přívodním
  potrubí osazen regulátor konstantního průtoku CAV a na odvodním potrubí
  regulátor proměnného průtoku VAV. Regulátor VAV na odtahu bude ovládán na
  základě konstantního podtlaku max. 50Pa v místnosti, popřípadě dle chodu
  odsávacích ventilátorů technologie. Otáčky ventilátorů v centrální jednotce
  budou řízeny na konstantní tlak v potrubí. 

}{
  Ventilation, cooling and heating of Central Grinding will be ensured by single
  AHU placed in AC machinery room on the 2nd floor. Thermally conditioned
  outdoor air will be supplied through ventilation duct to indoor areas by
  diffusers. Inlet diffusers will be controlled by 24~V servomotors according to
  room temperature. Waste air will be exhausted by ducts leading below the
  ceiling of ventilated area. 

  Since there will be a heat load from technology, inlet duct will
  be equipped with CAV controller ensuring constant air flow rate. Outlet duct
  will be equipped with VAV controller allowing varying air flow rate. This
  controller will be set in order to ensure constant underpressure in room, max.
  50~Pa, optionally depending on operation of exhaust fans of technology. Fan speeds in
  central unit will be controlled to ensure constant pressure in duct. 

}

\twoColls{
  Množství přiváděného vzduchu je navrženo na 4 výměny vzduchu za hodinu do výšky
  místnosti 3m. 

}{
  The amount of supplied air is designed to ensure 4 air exchanges per hour up
  to room height of 3 m.

}

\twoColls{
  Vzduchotechnická jednotka je ve vnitřním provedení, opatřena filtrem vzduchu
  (přívod F5+F7 /odtah F5), deskovým výměníkem s obtokem, vodním chladičem
  (15/20\gc), vodním chladičem (9/15\gc), vodním ohřívačem (45/30\gc) a dvěma
  ventilátory s frekvenčními měniči. 

  Zařízení pracuje s nuceným přívodem a odvodem
  vzduchu. Čerstvý vzduch je nasáván VZT jednotkou potrubím z fasády objektu, kde
  je umístěna protidešťová žaluzie. Znehodnocený vzduch je VZT jednotkou vyfukován
  nad střechu objektu.       

}{
  Innerly designed AHU will be equipped with air filters (F5+F7 supply / F5 exhaust),
  plate heat exchanger with bypass, water-based air cooler (15 /
  20\gc), water-based air cooler (9 / 15\gc) water-based air heater
  (45 / 30\gc) and two fans with frequency changers. 

  Equipment will work with forced air inlet and outlet. Fresh air will be
  suctioned in by AHU through ducts from building facade, where a weather
  resistant louver will be placed. Waste air will be blown out by AHU over the roof
  of the building.

}

\twoColls{
  Návrhové parametry jednotky, jsou uvedeny v příloze: TABLE EQUIPMENT AND
  PRINCIPLE DIAGRAMS.

}{
  Unit design parameters are listed in annex: TABLE EQUIPMENT AND PRINCIPLE
  DIAGRAMS.

}

\clearpage
\section{Equipment No.09.04-7	- MAF Quality Room}

\twoColls{
  Větrání, chlazení, vytápění, vlhčení a odvlhčování prostoru MAF Quality room
  zajišťuje soustava dvou vzduchotechnických jednotek a jedné adsorpční
  odvlhčovací jednotky. Všechny jednotky jsou osazeny ve strojovně vzduchotechniky
  v 2NP. Jednotka MAU a adsorpční jednotka slouží pro úpravu venkovního vzduchu o
  objemu 30\% z celkového množství vzduchu přiváděného do Quality room.

  Jednotka AHU je pouze cirkulační a přivádí teplotně a vlhkostně upravovaný
  vzduch potrubím do klimatizovaných prostor, kam je vyfukován pomocí vířivých
  anemostatů. 
  Odvod vzduchu je pomocí odsávací potrubí lokálně vyústěného
  v jednotlivých místnostech. Množství přiváděného vzduchu je navrženo v místnosti
  MAF Quality room, dle potřeby odvodu tepelné zátěže a v ostatních prostorách,
  dle standardu LEGO. 

  Vzduchotechnická jednotka AHU, je ve vnitřním provedení,
  opatřena filtrem vzduchu F7, vodním chladičem (9/15\gc), vodním ohřívačem
  (45/30\gc), přívodním ventilátorem s frekvenčním měničem a zvlhčovací komorou. Do
  komory je osazena parní trubice, napojená na lokální elektrický odporový vyvíječ
  páry.
}{
  Ventilation, cooling, heating, humidification and dehumidification of MAF
  Quality Room will be ensured by two AHUs and one adsorption dehumidifying
  unit. All units will be placed in AC machinery room on the 2nd floor. MAU and
  adsorption unit will condition 30~\% of total amount of air supplied to
  Quality room.

  AHU will be circulation only and will supply thermally conditioned and
  humidity regulated air through a duct to climatized areas, where it will be
  blown by swirl diffusers. Waste air will be suctioned out through a duct initiating in
  each individual room.  Amount of supplied air will be designed in the MAF
  Quality room in order to lead away enough heat load to comply with LEGO Standards in
  the remaining areas.

  Innerly designed AHU will be equipped with F7 air filter, water-based air
  cooler (9 / 15\gc), water-based air heater (45 / 30\gc), supply fan with
  frequency changer and humidification chamber. This chamber will be equipped
  with steam pipe connected to a local electrical resistance-based steam
  producer.  

}

\twoColls{
  Vzduchotechnická jednotka MAU, je ve vnitřním provedení, opatřena filtrem
  vzduchu (přívod F5 / odtah F5), rotačním výměníkem, vodním chladičem (9/15\gc),
  vodním ohřívačem (45/30\gc) a dvěma ventilátory s frekvenčními měniči.

}{
  Innerly designed MAU will be equipped with air filters (F5 suppy / F5
  exhaust), rotatory heat exchanger, water-based air cooler (9 / 15\gc),
  water-base air heater (45 / 30\gc) and two fans with frequency changers. 

}

\twoColls{
  Vzduchotechnická adsorpční jednotka, je ve vnitřním provedení, opatřena filtrem
  vzduchu (přívod F5 / odtah F5), rotačním adsorpčním výměníkem, elektrickým
  ohřívačem, a dvěma ventilátory s fre\-kven\-ční\-mi měniči.

}{
  Innerly designed air conditioning adsorption unit will be equipped with air
  filters (F5 supply / F5 exhaust), rotatory adsorption exchanger, electrical air
  heater and two fans with frequency changers. 

}

\twoColls{
  V dalším stupni dokumentace, je uvažováno se sloučením MAU jednotky a adsorpční
  jednotky do jedné jednotky od jednoho výrobce.

}{
  In next stage of documentation, MAU and adsorption unit may be fused into one
  unit from one manufacturer. 

}

\twoColls{
  Jelikož jsou jedním centrálním zařízením klimatizovány tři prostory, jsou
  v přívodním potrubí osazeny regulátory průtoku VAV s vodními ohřívači a
  v odvodním potrubí regulátory průtoku VAV. Regulátory průtoku VAV s ohřívači na
  přívodu budou regulovány dle prostorového teplotního čidla. 
  Regulátory VAV na odvodu budou regulovány souběžně s regulátory VAV na přívodu.

  Otáčky ventilátorů v centrálních jednotkách budou řízeny na konstantní tlak
  v potrubí. Parní výkon zvlhčování bude řízen dle prostorového čidla relativní
  vlhkosti v místnosti MAF Quality room.

}{
  Since 3 areas will be climatized by one central equipment, inlets will be
  equipped with flow rate VAV controllers with water-based air heaters whereas
  outlets will be equipped with flow rate VAV controllers only.
  Whole equipment in inlets will be controlled according to temperature room sensor.
  As soon as these controllers are set, controller settings on outlets will be
  determined. 

  Fan speeds in central AHUs will be set to ensure constant pressure in supply and
  exhaust ducts. Power of steam humidification will be controlled by relative
  humidity sensor placed in MAF Quality Room.

}

\twoColls{
  Návrhové parametry jednotek a ventilátorů, jsou uvedeny v příloze: TABLE
  EQUIPMENT AND PRINCIPLE DIAGRAMS.

}{
  Units and fans design parameters are listed in annex: TABLE EQUIPMENT AND
  PRINCIPLE DIAGRAMS.

}

\clearpage
\section{Equipment No.09.04-8	- MAF Area}

\twoColls{
  Větrání, chlazení a vytápění MAF Area zajišťuje jedna vzduchotechnická jednotka
  osazená ve strojovně vzduchotechniky v 2NP.  Teplotně upravovaný venkovní vzduch
  je přiváděn potrubím do prostoru, kam je vyfukován pomocí vyústek. Přívodní
  vyústky v MAF Area  jsou ovládány servomotory na 24V, dle prostorvé teploty.
  Odvod vzduchu je pomocí odsávacího potrubí, vedeného pod stropem větraného
  prostoru. Do místností MAF Grinding a MAF Inbound Goods je přiváděný vzduch je
  vyfukován pomocí vířivých anemostatů.

  Jelikož jsou v místnosti MAF Area technologický odtahy, bude na přívodním
  potrubí osazen regulátor konstantního průtoku CAV a na odvodním potrubí
  regulátor proměnného průtoku VAV. Regulátor VAV na odtahu bude ovládán na
  základě konstantního podtlaku max. 50Pa v místnosti, popřípadě dle chodu
  odsávacích ventilátorů technologie. 

  Jelikož jsou jedním centrálním zařízením větrány další dva prostory, jsou
  v přívodním potrubí pro MAF Grinding a MAF Inbound Goods osazeny regulátory
  průtoku VAV s vodními ohřívači a v odvodním potrubí regulátory průtoku VAV.
  Regulátory průtoku VAV s ohřívači na přívodu budou regulovány dle prostorového
  teplotního čidla.  Regulátory VAV na odvodu budou regulovány souběžně
  s regulátory VAV na přívodu.  Otáčky ventilátorů v centrálních jednotkách budou
  řízeny na konstantní tlak v potrubí.

}{
  Ventilation, cooling and heating of MAF Area will be ensured by single
  AHU placed in AC machinery room on the 2nd floor. Thermally conditioned
  outdoor air will be supplied through ventilation duct to indoor areas by
  diffusers. Inlet diffusers will be controlled by 24~V servomotors according to
  room temperature. Waste air will be suctioned out through ducts leading below the
  ceiling of ventilated area. Air will be supplied to MAF Grinding and MAF
  Inbound Goods rooms through swirl diffusers.

  Since there will be a heat load from technology, inlet duct will
  be equipped with CAV controller ensuring constant air flow rate. Outlet duct
  will be equipped with VAV controller allowing varying air flow rate. This
  controller will be set in order to ensure constant underpressure in room, max.
  50~Pa, optionally based on operation of exhaust fans of technology. 

  Because two other areas will be ventilated by one central unit, inlets for MAF
  Grinding and MAF Inbound Goods rooms will be equipped with flow rate VAV
  controllers with water-based air heaters whereas outlets will be equipped with
  flow rate VAV controllers only.  Whole equipment in inlets will be controlled
  according to temperature room sensor.  As soon as these controllers are set,
  controller settings on outlets will be determined.  Fan speeds in central unit
  will be controlled to ensure constant pressure in duct.  

}

\twoColls{
  Množství přiváděného vzduchu je navrženo na 4 výměny vzduchu za hodinu do výšky
  místnosti 3m. 

}{
  The amount of supplied air is designed to ensure 4 air exchanges per hour up
  to room height of 3 m.

}

\twoColls{
  Vzduchotechnická jednotka je ve vnitřním provedení, opatřena filtrem vzduchu
  (přívod F5+F7 /odtah F5), rotačním výměníkem, směšovací komorou, vodním
  chladičem (15/20\gc), vodním chladičem (9/15\gc), vodním ohřívačem (45/30\gc) a
  dvěma ventilátory s frekvenčními měniči. 

  Zařízení pracuje s nuceným přívodem a
  odvodem vzduchu. Čerstvý vzduch je nasáván VZT jednotkou potrubím z fasády
  objektu, kde je umístěna protidešťová žaluzie. Znehodnocený vzduch je VZT
  jednotkou vyfukován nad střechu objektu.       

}{
  Innerly designed AHU will be equipped with air filters (F5+F7 supply / F5 exhaust),
  rotatory heat exchanger, mixing chamber, water-based air cooler (15 /
  20\gc), water-based air cooler (9 / 15\gc) water-based air heater
  (45 / 30\gc) and two fans with frequency changers. 

  Equipment will work with forced air
  inlet and outlet. Fresh air will be suctioned in by AHU through ducts from building
  facade, where a weather resistant louver will be installed. 
  Waste air will be blown
  by AHU over the roof of the building.

}

\twoColls{
  Cirkulační klapka ve směšovací komoře bude v běžném provozu trvale zavřená a
  jednotka bude pracovat pouze s čerstvým vzduchem. Při odstávce technologie
  v zimním období, pokud bude třeba vytápět vnitřní prostor, budou klapky ve
  směšovací komoře přepnuty do 100\% cirkulace a odsávací ventilátor bude vypnut. 

}{
  During standard operation, circulation valve in mixing chamber will be closed
  and the unit will work in mode with fresh air only. During a technology
  shutdown in winter season, when there will be a necessity to heat indoor
  areas, valves in mixing chamber will be switched to 100~\% circulation mode
  and exhaust fan will be turned off.

}

\twoColls{
  Návrhové parametry jednotky, jsou uvedeny v příloze: TABLE EQUIPMENT AND
  PRINCIPLE DIAGRAMS.

}{
  Unit design parameters are listed in annex: TABLE EQUIPMENT AND PRINCIPLE
  DIAGRAMS.

}

\clearpage
\section{Equipment No.09.04-9	- MAF + MQ Office}

\twoColls{
  Větrání prostor kanceláří a jednacích místnosti zajišťuje jedna vzduchotechnická
  jednotka osazená ve strojovně vzduchotechniky v 2NP. Teplotně upravovaný
  venkovní vzduch je přiváděn potrubím do vnitřních prostor pomocí vířivých
  anemostatů, jež jsou osazeny do podhledu, a znehodnocený vzduch je odsáván
  potrubím vedeným nad podhledem. Propojení mezi větranou místností a odsávaným
  podhledem, bude zajištěno mřížovanou podhledovou kazetou. 

  Pro každou větranou zónu (kancelář, zasedací místnost) budou do přívodního
  potrubí osazeny regulátory průtoku VAV s vodními ohřívači a do odvodního potrubí
  regulátory průtoku VAV. Regulátory průtoku VAV s ohřívači na přívodu budou
  regulovány dle prostorového teplotního čidla z každé místnosti. Regulátory VAV
  na odvodu budou regulovány souběžně s regulátory VAV na přívodu. Otáčky
  ventilátorů v centrálních jednotkách budou řízeny na konstantní tlak v potrubí.

}{

  Office areas and meeting rooms will be ventilated by single AHU which will be
  placed in AC machinery room on the 2nd floor.  Thermally conditioned outdoor air
  will be supplied through ventilation duct to indoor areas by swirl diffusers,
  which will be installed in ceiling.  Waste air will be exhausted by ducts
  leading above the ceiling. Ventilated office will be interconnected with
  exhausting ceiling by reticulated cartridges mounted in the ceiling. 

  For each ventilated area (office or meeting room), supply ducts will be equipped
  with flow rate VAV controllers with water-based air heaters whereas the exhaust
  ducts will be equipped with flow rate VAV controllers only. Whole equipment in
  inlets will be controlled according to temperature sensor in each room. As soon
  as these controllers are set, controller settings on outlets will be determined.
  Fan speeds in central AHUs will be set to ensure constant pressure in supply and
  exhaust ducts.

}

\twoColls{
  V každé místnosti bude umístěno čidlo kvality vzduchu, a poměr směšování
  v centrální VZT jednotce, bude řízen dle nejhorší hodnoty ze všech čidel.
  Přiváděná dávka venkovního vzduchu do kanceláře je 50m3/h na osobu a do zasedací
  místnosti 80m3/h na osobu, nebo min. 4x/h.

}{
  Air quality sensor will be placed in each room and according to the worse
  sensor value, the mixing ration of fresh air will be set in central AHU. The
  amount of outdoor air supplied to offices is calculated for 50~m3/h per person
  and for meeting rooms for 80~m3/h per person. There will always be at least 4
  air exchanges per hour.

}

\twoColls{
  Vzduchotechnická jednotka je ve vnitřním provedení, opatřena filtrem vzduchu
  (přívod F5+F7 /odtah F5), rotačním výměníkem, směšovací komorou, vodním
  chladičem (15/20\gc), vodním chladičem (9/15\gc), vodním ohřívačem (45/30\gc) a
  dvěma ventilátory s frekvenčními měniči. 

  Zařízení pracuje s nuceným přívodem a
  odvodem vzduchu. Čerstvý vzduch je nasáván VZT jednotkou potrubím z fasády
  objektu, kde je umístěna protidešťová žaluzie. Znehodnocený vzduch je VZT
  jednotkou vyfukován nad střechu objektu.       

}{
  Innerly designed AHU will be equipped with air filters (F5+F7 supply / F5
  exhaust), rotatory heat exchanger, mixing chamber, water cooler (15 / 20\gc),
  water cooler (9 / 15\gc), water heater (45 / 30\gc) and two fans with frequency
  changers.

  Equipment will work with forced air inlet and outlet. Fresh air will be
  suctioned in by AHU through ducts from building facade where a weather
  resistant louver will be placed. Waste air will be blown by AHU over the roof
  of the building.

}

\twoColls{
  Chlazení kanceláří je zajištěno pomocí centrální VZT jednotky (VAV systém).
  Maximální přiváděné množství vzduchu do kanceláře a zasedací místnosti je
  navrženo na odvod letní tepelné zátěže, snížení množství přívodního vzduchu
  zajistí minimální hygienické větrání. V zasedací místnosti bude zajištěna
  možnost zcela uzavřít přívod a odvod vzduchu (zajistí profese MaR) tlačítkem
  z prostoru místnosti. 

}{
  Offices cooling will be ensured by central AHU (variable air volume system –
  VAV). Maximal amount of air supplied to offices and meeting room is designed to
  remove heat gains in summer time. Amount of supplied air will always comply with
  a requirement of minimal hygienic ventilation.  In meeting room, there is a
  possibility of complete closure of the ventilation systems by the switches
  (ensured by profession M\&C – Measurement and Control). 

}

\twoColls{
  Návrhové parametry jednotky, jsou uvedeny v příloze: TABLE EQUIPMENT AND
  PRINCIPLE DIAGRAMS.

}{
  Unit design parameters are listed in annex: TABLE EQUIPMENT AND PRINCIPLE
  DIAGRAMS.

}

\clearpage
\section{Equipment No.09.04-10 - Mould Qualification}

\twoColls{
  Větrání, chlazení a vytápění prostor Mould Qualification zajišťují dvě
  vzduchotechnické jednotky osazené ve strojovně vzduchotechniky v 2NP. Jednotka
  poz. 10.1 zajišťuje větrání velkého prostoru Molud Qualification, kde jsou
  umístěny lisy. Jednotka poz. 10.2 zajišťuje větrání místností MQ Workshop a MQ
  Material room náležících k Molud Qualification.

}{
  Ventilation, cooling and heating of Mould Qualification will be ensured by two
  AHUs placed in AC machinery room on the 2nd floor. Unit ref. 10.1 will ensure
  ventilation of overall area of Mould Qualification, where moulding machines
  will be placed. Unit ref. 10.2 will ensure ventilation of MQ Workshop and MQ
  Material rooms belonging to Mould Qualification.

}

\twoColls{
  Jednotkou poz. 10.1 je teplotně upravovaný venkovní vzduch přiváděn potrubím do
  prostoru, kam je vyfukován pomocí vyústek. Přívodní vyústky umístěné v Mould
  Qualification jsou ovládány servomotory na 24V, dle prostorvé teploty.  Odvod
  vzduchu je pomocí odsávacího potrubí, vedeného pod stropem větraného prostoru. 

}{
  Unit ref. 10.1 will supply thermally conditioned air through ventilation duct
  to indoor areas, where it will be blown by diffusers. Inlet diffusers will be
  controlled by 24~V servomotors according to room temperature. Waste air will
  be exhausted by exhaust ducts leading below the ceiling of ventilated area.

}

\twoColls{
  Množství přiváděného vzduchu je navrženo dle potřeby odvodu tepelné zátěže od
  technologie.

}{
  The amount of supplied air is designed to ensure exhaust of heat load from
  technology.

}

\twoColls{
  Vzduchotechnická jednotka je ve vnitřním provedení, opatřena filtrem vzduchu
  (přívod F5+F7 /odtah F5), rotačním výměníkem, směšovací komorou, vodním
  chladičem (15/20\gc), vodním chladičem (9/15\gc), vodním ohřívačem (45/30\gc) a
  dvěma ventilátory s frekvenčními měniči. 

  Zařízení pracuje s nuceným přívodem a
  odvodem vzduchu. Čerstvý vzduch je nasáván VZT jednotkou potrubím z fasády
  objektu, kde je umístěna protidešťová žaluzie. Znehodnocený vzduch je VZT
  jednotkou vyfukován nad střechu objektu.       

}{
  Innerly designed AHU will be equipped with air filters (F5+F7 supply / F5 exhaust),
  rotatory heat exchanger, mixing chamber, water-based air cooler (15 /
  20\gc), water-based air cooler (9 / 15\gc) water-based air heater
  (45 / 30\gc) and two fans with frequency changers. 

  Equipment will work with forced air inlet and outlet. Fresh air will be
  suctioned in by AHU through ducts from building facade, where a weather
  resistant louver will be installed. Waste air will be blown by AHU over the roof
  of the building.

}

\twoColls{
  Návrhové parametry jednotky, jsou uvedeny v příloze: TABLE EQUIPMENT AND
  PRINCIPLE DIAGRAMS.

}{
  Unit design parameters are listed in annex: TABLE EQUIPMENT AND PRINCIPLE
  DIAGRAMS.

}

\twoColls{
  Jednotkou poz.10.2 jsou větrány místnosti MQ Workshop a MQ Material room.
  Teplotně upravovaný venkovní vzduch je přiváděn potrubím do prostoru, kam je
  vyfukován pomocí vyústek. Přívodní vyústky umístěné v  MQ Workshop jsou ovládány
  servomotory na 24V, dle prostorvé teploty.  Odvod vzduchu je pomocí odsávacího
  potrubí, vedeného pod stropem větraného prostoru. Jelikož je jedním zařízení
  větráno více prostor a dále jsou zde umístěny technologické odtahy budou na
  přívodním a odvodním potrubí osazeny regulátory proměnného průtoku VAV. 

}{
  Unit ref. 10.2 will ensure ventilation of MQ Workshop and MQ Material rooms.
  Thermally conditioned outdoor air will be forced through ventilation duct to
  indoor areas, where it will be blown out through diffusers. Inlet diffusers
  will be controlled by 24~V servomotors according to temperature in each room.
  Waste air will be exhausted through an exhaust duct leading below the ceiling
  of ventilated area.  Since many areas will be ventilated by one central
  equipment, which includes the exhaust of heat load from technology, both,
  inlets and outlets, will be equipped with flow rate VAV controllers.

}

\twoColls{
  Regulátor průtoku VAV v MQ Material room bude regulován dle prostorového
  teplotního čidla. Regulátory VAV na odvodu budou regulovány souběžně
  s regulátory VAV na přívodu. 

}{
  VAV controllers in inlets in MQ Material room will be set according to
  temperature room sensor. As soon as these controllers are set,
  controller settings on outlets will be determined.

}

\twoColls{
  Jelikož jsou v místnostech lokální odtahy od technologie, bude VAV regulátor na
  odtahu ovládán ještě na základě konstantního podtlaku max. 50Pa v místnosti,
  popřípadě dle chodu odsávacích ventilátorů technologie. Otáčky ventilátorů
  v centrální jednotce budou řízeny na konstantní tlak v potrubí. 

}{
  Since there will be a heat load from technology, VAV controller in outlet duct
  will be set in order to ensure constant underpressure in room, max. 50~Pa,
  optionally depending on operation of exhaust fans of technology. Fan speeds in
  central unit will be controlled to ensure constant pressure in duct. 

}

\twoColls{
  Množství přiváděného vzduchu je navrženo dle standardu LEGO.  

}{
  The amount of supplied air is designed to comply with LEGO standard.

}

\twoColls{
  Vzduchotechnická jednotka je ve vnitřním provedení, opatřena filtrem vzduchu
  (přívod F5+F7 /odtah F5), rotačním výměníkem, směšovací komorou, vodním
  chladičem (15/20\gc), vodním chladičem (9/15\gc), vodním ohřívačem (45/30\gc) a
  dvěma ventilátory s frekvenčními měniči. 

  Zařízení pracuje s nuceným přívodem a
  odvodem vzduchu. Čerstvý vzduch je nasáván VZT jednotkou potrubím z fasády
  objektu, kde je umístěna protidešťová žaluzie. Znehodnocený vzduch je VZT
  jednotkou vyfukován nad střechu objektu.       

}{
  Innerly designed AHU will be equipped with air filters (F5+F7 supply / F5
  exhaust), rotatory heat exchanger, mixing chamber, water-based air cooler (15
  / 20\gc), water-based air cooler (9 / 15\gc) water-based air heater (45 /
  30\gc) and two fans with frequency changers. 

  Equipment will work with forced air inlet and outlet.  Fresh air will be
  suctioned in by AHU through ducts from building facade, where a weather
  resistant louver will be installed. Waste air will be blown by AHU over the roof
  of the building.

}

\twoColls{
  Návrhové parametry jednotky, jsou uvedeny v příloze: TABLE EQUIPMENT AND
  PRINCIPLE DIAGRAMS.

}{
  Unit design parameters are listed in annex: TABLE EQUIPMENT AND PRINCIPLE
  DIAGRAMS.

}

\clearpage
\section{Equipment No.09.04-11 - Aisle I.II.III.IV}

\twoColls{
  Větrání, chlazení a vytápění prostor chodeb zajišťuje jedna vzduchotechnická
  jednotka osazená ve strojovně vzduchotechniky ve 2NP. Teplotně upravovaný
  venkovní vzduch je přiváděn potrubím do prostoru, kam je vyfukován pomocí
  vyústek. Odvod vzduchu je pomocí odsávacího potrubí, vedeného pod stropem
  větraného prostoru.

}{
  Ventilation, cooling and heating of corridors will be ensured by single AHU
  placed in AC machinery room on the 2nd floor. Thermally conditioned outdoor air
  will be forced through ventilation duct to corridors, where it will be blown out
  through diffusers. Waste air will be suctioned out through exhaust ducts leading below
  the ceiling of ventilated area.

}

\twoColls{
  Množství přiváděného vzduchu je navrženo na 1 výměnu vzduchu za hodinu na celou
  výšku místnosti.

}{
  The amount of supplied air is designed to ensure 1 air exchange per hour 
  throughout room height. 

}

\twoColls{
  Vzduchotechnická jednotka je ve vnitřním provedení, opatřena filtrem vzduchu
  (přívod F5+F7 /odtah F5), rotačním výměníkem, směšovací komorou, vodním
  chladičem (15 / 20\gc), vodním ohřívačem (45 / 30\gc) a dvěma ventilátory
  s frekvenčními měniči. 

  Zařízení pracuje s nuceným přívodem a odvodem vzduchu.
  Čerstvý vzduch je nasáván VZT jednotkou potrubím z fasády objektu, kde je
  umístěna protidešťová žaluzie. Znehodnocený vzduch je VZT jednotkou vyfukován
  nad střechu objektu.       

}{
  Innerly designed AHU will be equipped with air filters (F5+F7 supply / F5
  exhaust), rotatory heat exchanger, mixing chamber, water cooler (15 / 20\gc),
  water heater (45 / 30\gc) and two fans with frequency changers. 

  Equipment will work with forced air inlet and outlet. Fresh air will be
  suctioned in by AHU through ducts from building facade, where a weather
  resistant louver will be installed. Waste air will be blown by AHU over the
  roof of the building.

}

\twoColls{
  Návrhové parametry jednotky, jsou uvedeny v příloze: TABLE EQUIPMENT AND
  PRINCIPLE DIAGRAMS.

}{
  Unit design parameters are listed in annex: TABLE EQUIPMENT AND PRINCIPLE
  DIAGRAMS.

}

\clearpage
\section{Equipment No.09.04-12 - Protected escape route}

\twoColls{
  Větrání dvou chráněných únikových cest (CHÚC) je řešeno nuceným přetlakovým
  způsobem. Pro každou CHÚC  je navrženo samostatné zařízení. Venkovní vzduch je
  nasáván potrubním ventilátorem nad střechou objektu a pomocí potrubí, je
  přiveden do prostoru CHÚC. Přívodní potrubí vedené jiným požárním úsekem, bude
  požárně izolováno.

}{
  Two protected escape routes (PER) will be ventilated by a forced overpressure way.
  For each separate PER an individual equipment was devised. Outdoor air will be
  suctioned in by a duct ventilator located above the roof of the building and led through
  a duct to the areas of PER. Supply duct leading through fire compartments
  will be provided with a fire insulation. 

}

\twoColls{
  Odvod vzduchu z větraného prostoru je zajištěn vlivem přetlaku v nejvyšším místě
  CHÚC. V odvodním potrubí bude osazena uzavírací a přetlaková klapka, na které
  bude nastaven požadovaný přetlak cca 25Pa. Celkem budou použity 3ks výfukových
  kompletů. Přiváděné množství vzduchu do CHÚC zajistí 10 výměn vzduchu za hodinu.

}{
  Air exhaust from ventilated area will be ensured by overpressure in the highest
  point of PER. The outlet will be provided with an overpressure valve on its inner
  side, which will be set to ensure the overpressure of ca. 25~Pa. A total number of
  three exhaust sets will be installed. 
  The amount of supplied air is designed to ensure 10 air exchanges per hour.

}

\twoColls{
  Návrhové parametry ventilátoru, jsou uvedeny v příloze: TABLE EQUIPMENT AND
  PRINCIPLE DIAGRAMS.

}{
  Fan design parameters are listed in annex: TABLE EQUIPMENT AND PRINCIPLE
  DIAGRAMS.

}

\clearpage
\section{Equipment No.09.04-13 - Sanitary room}

\twoColls{
  Hygienické místnosti, jsou větrány podtlakovým způsobem pomocí samostatného
  odsávacího zařízení složeného z potrubí, odsávacího ventilátoru umístěného pod
  stropem větrané místnosti a přetlakové klapky. Znehodnocený vzduch je
  z jednotlivých místností odsáván pomocí vyústek osazených do podhledu a je
  ventilátorem vyfukován na střechu objektu. Náhradní vzduch je přisáván přes
  stěnové mřížky a pod dveřmi z prostoru chodby.

}{
  Sanitary rooms will be ventilated by an underpressure way by an usage of
  separate suction device composed of ducts, exhaust fan installed below the
  ceiling of ventilated room and overpressure valve. Waste air will be
  suctioned out from individual rooms with diffusers mounted in the ceiling and
  blown by ventilator over the roof of the building. Replacement air will be
  suctioned in through wall grilles and from corridor through space under the door.

}

\twoColls{
  Návrhové parametry ventilátoru, jsou uvedeny v příloze: TABLE EQUIPMENT AND
  PRINCIPLE DIAGRAMS.

}{
  Fan design parameters are listed in annex: TABLE EQUIPMENT AND PRINCIPLE
  DIAGRAMS.

}
